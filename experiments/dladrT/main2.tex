\documentclass[10pt,a4paper,onecolumn]{article}
\usepackage[english,polish]{babel}

% Użyj polskiego łamania wyrazów (zamiast domyślnego angielskiego).
\usepackage{polski}

\usepackage[utf8]{inputenc}

% dodatkowe pakiety

\usepackage{mathtools}
\usepackage{amsfonts}
\usepackage{amsmath}
\usepackage{amsthm}

%\DeclareMathOperator{\sgn}{sgn}

\usepackage{csquotes}
% Ponieważ `csquotes` nie posiada polskiego stylu, można skorzystać z mocno zbliżonego stylu chorwackiego.
\DeclareQuoteAlias{croatian}{polish}

\AtBeginDocument{
	\renewcommand{\tablename}{Tabela}
	\renewcommand{\figurename}{Fig.}
}

% ------------------
% --- < tabele > ---

\usepackage{array}
\usepackage{tabularx}
\usepackage{multirow}
\usepackage{booktabs}
\usepackage{makecell}
\usepackage[flushleft]{threeparttable}
%\usepackage{longtable} %????
\usepackage{longtable}

% defines the X column to use m (\parbox[c]) instead of p (`parbox[t]`)
\newcolumntype{C}[1]{>{\hsize=#1\hsize\centering\arraybackslash}X}

% ------------------------------
% --- < linki i referencje > ---

\usepackage[hidelinks, colorlinks=false]{hyperref}
\usepackage{cleveref}

% ------------------
% --- < figury > ---

% najprawdopodobniej jakaś paczka ładuje subcaption, a svg ładuje subfig, które
% nie są kompatybilne; poniższe rozwiązanie wzięte z https://tex.stackexchange.com/a/213279
\usepackage{subcaption}
\expandafter\def\csname ver@subfig.sty\endcsname{}


\usepackage{graphicx}
\usepackage{float} % lepsze pozycjonowanie
\usepackage{svg}
\graphicspath{{graphics/}}  % obrazki i zdjęcia muszą być w podfolderze "graphics"

% ----------------------------
% --- < liczby i symbole > ---

\usepackage{siunitx}

\sisetup{
    group-digits = integer,
    list-final-separator = { \translate{and} },
    range-phrase = { -- },
    mode = text,
}

% defines the X column to use m (\parbox[c]) instead of p (`parbox[t]`)
%\newcolumntype{C}[1]{>{\hsize=#1\hsize\centering\arraybackslash}X}

\DeclareMathOperator{\sgn}{sgn}

\begin{document}

Wyprowadzenie wzorów zupełnie jak w Pana mailu pt. ,,Identyfikacja momentu bezwładności kulki'' z 26.05.2017.

Siły działające na kulkę: $F_1$ -- składowa siły ciężkości działająca równolegle do pochylni, $N_1$ -- składowa reakcji podłoża również działająca równolegle, ale skierowana przeciwnie do $F_1$.

\[
F_1 = m g \sin \alpha
\]

\[
N_1 = \frac{\epsilon J}{r}
\]

Zatem wypadkowa siła działająca na kulkę to $F_1 - N_1 = m g \sin \alpha - \frac{\epsilon J}{r} = a m$, skąd można otrzymać wzór na $a$, po skorzystaniu z zależności $\epsilon r = a$:

\begin{align*}
    a m &= m g \sin \alpha - \frac{\epsilon J}{r} \\
    a m &= m g \sin \alpha - \frac{a J}{r^2} \\
    a (m + \frac{J}{r^2}) &= mg\sin\alpha \\
    a &= \frac{mg\sin\alpha}{m + \frac{J}{r^2}} \\
    a &= \frac{mgr^2\sin\alpha}{mr^2 + J}
\end{align*}

Wzór na moment bezwładności kulki $J = \frac{2}{5} m r^2$, dalej:

\begin{align*}
    a &= \frac{mgr^2\sin\alpha}{mr^2 + J} \\
    a &= \frac{mgr^2\sin\alpha}{mr^2 + \frac{2}{5}mr^2} \\
    a &= \frac{g\sin\alpha}{1+\frac{2}{5}} \\
    a &= \frac{5g\sin\alpha}{7} \\
    a &\approx 7\sin\alpha \approx 7 \alpha
\end{align*}

\end{document}