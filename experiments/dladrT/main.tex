\documentclass[10pt,a4paper,onecolumn]{article}
\usepackage[english,polish]{babel}

% Użyj polskiego łamania wyrazów (zamiast domyślnego angielskiego).
\usepackage{polski}

\usepackage[utf8]{inputenc}

% dodatkowe pakiety

\usepackage{mathtools}
\usepackage{amsfonts}
\usepackage{amsmath}
\usepackage{amsthm}

\usepackage{csquotes}
% Ponieważ `csquotes` nie posiada polskiego stylu, można skorzystać z mocno zbliżonego stylu chorwackiego.
\DeclareQuoteAlias{croatian}{polish}

\AtBeginDocument{
	\renewcommand{\tablename}{Tabela}
	\renewcommand{\figurename}{Fig.}
}

% ------------------
% --- < tabele > ---

\usepackage{array}
\usepackage{tabularx}
\usepackage{multirow}
\usepackage{booktabs}
\usepackage{makecell}
\usepackage[flushleft]{threeparttable}
%\usepackage{longtable} %????
\usepackage{longtable}

% defines the X column to use m (\parbox[c]) instead of p (`parbox[t]`)
\newcolumntype{C}[1]{>{\hsize=#1\hsize\centering\arraybackslash}X}

% ------------------------------
% --- < linki i referencje > ---

\usepackage[hidelinks, colorlinks=false]{hyperref}
\usepackage{cleveref}

% ------------------
% --- < figury > ---

% najprawdopodobniej jakaś paczka ładuje subcaption, a svg ładuje subfig, które
% nie są kompatybilne; poniższe rozwiązanie wzięte z https://tex.stackexchange.com/a/213279
\usepackage{subcaption}
\expandafter\def\csname ver@subfig.sty\endcsname{}


\usepackage{graphicx}
\usepackage{float} % lepsze pozycjonowanie
\usepackage{svg}
\graphicspath{{graphics/}}  % obrazki i zdjęcia muszą być w podfolderze "graphics"

% ----------------------------
% --- < liczby i symbole > ---

\usepackage{siunitx}

\sisetup{
    group-digits = integer,
    list-final-separator = { \translate{and} },
    range-phrase = { -- },
    mode = text,
}

% defines the X column to use m (\parbox[c]) instead of p (`parbox[t]`)
%\newcolumntype{C}[1]{>{\hsize=#1\hsize\centering\arraybackslash}X}

\DeclareMathOperator{\sgn}{sgn}

\begin{document}



%%%%%%%%%%%%%%%%%%%%%%%%%%%%%%%%%%%%%%%%%%%%%

\begin{table}[h]
    \centering
    \begin{threeparttable}
        \caption{Parametry fizyczne, elektryczne i mechaniczne silnika, enkodera i przekładni\tnote{a}.}
        \label{tab:parametry_silnika}
        
        \begin{tabularx}{1\textwidth}{l | l}
            \toprule
            Średnica & \SI{37}{\milli\meter} \\
            Długość & \SI{68}{\milli\meter} \\
            Masa & \SI{215}{g} \\
            Średnica wału & \SI{6}{\milli\meter} \\
            \midrule
            Przełożenie przekładni $f$ & \num{18,75}:\num{1} \\
            \midrule
            Napięcie znamionowe $u_N$ & \SI{12}{\volt} \\
            Prędkość znamionowa $\omega_N$ & \SI{52,36}{\radian\per\second} \\
            Prąd znamionowy $i_N$ & \SI{300}{\milli\ampere} \\
            Prąd zatrzymania silnika $i_S$ & \SI{5000}{\milli\ampere} \\
            Moment zatrzymania silnika $T_S$ & \SI{0,59}{\newton\meter} \\
            \midrule
            Rezystancja\tnote{b} $R$ & \SI{2,4}{\ohm} \\
            Stała SEM rotacji\tnote{b} $K_e$ & \SI{0,2154}{\volt\per\radian\per\second} \\
            Stała momentu\tnote{b} $K_t$ & \SI{0,2154}{\newton\meter\per\ampere} \\
            Współczynnik tarcia wiskotycznego\tnote{b} $\beta$ & \num{0,00161} \\
            Współczynnik tarcia suchego\tnote{b} $b$ & \num{0,019} \\
            Moment bezwładności przekładni i wału\tnote{c} $J$ & \num{0,00123} \\
            \bottomrule
        \end{tabularx}
        
        \begin{tablenotes}
            \footnotesize
            \item[a] opracowanie własne na podstawie ...,
            % TODO: zaktualizuj odnośniki
            \item[b] zidentyfikowano analitycznie, szczegóły poniżej,
            \item[b] zidentyfikowano eksperymentalnie, szczegóły poniżej.
        \end{tablenotes}
    \end{threeparttable}
\end{table}

Podstawowe równanie obwodu silnika (przyjęto zerową induktancję, $u_M = u \cdot u_N$ oraz $K = K_e = K_t$):
\begin{equation}\label{eq1}
u_M = R i + K \omega
\end{equation}

Podstawowe równanie momentu silnika:
\begin{equation}\label{eq2}
T = K_t i - J \dot{\omega} - \beta \omega - b \sgn \omega
\end{equation}

Rezystancja obwodów silnika została obliczona dla sytuacji zatrzymania silnika (równanie \ref{eq1}, $\omega = 0$), kiedy mamy:
\begin{equation}\label{eq3}
R = \frac{u \cdot u_N}{i_S} = 2,4 [\Omega]
\end{equation}

Z równania \ref{eq1} wynika, że przy prędkości jałowej:
\begin{equation}\label{eq4}
K = \frac{u \cdot u_N - R i_N}{\omega_N}
\end{equation}
co daje wartości $K_e = K_t = 0,2154$.


Współczynniki $b$ oraz $\beta$ zostały obliczone ze wzoru \ref{eq2}, przy ustalonej wartości prędkości obrotowej wału ($J\dot{\omega} = 0$) oraz przy ominięciu składnika momentu ($T = 0$):
\begin{equation}
K i = \beta \omega + b \sgn \omega 
\end{equation}
\begin{equation}
\frac{K}{R}(u \cdot u_N - K \omega) = \beta \omega + b \sgn \omega
\end{equation}
\begin{equation}
\frac{K u_N}{R} u = (\frac{K^2}{R} + \beta) \omega + b \sgn \omega
\end{equation}
\begin{equation}
\omega = \frac{K u_N}{R( \frac{K^2}{R} + \beta )}u - \frac{b}{\frac{K^2}{R} + \beta} \sgn \omega
\end{equation}

Stąd, wyznaczając współczynniki $K_1$, $K_2$ regresji liniowej $y = K_1 x + K_2$ możemy otrzymać wzory na $\beta$ oraz $b$:

\begin{equation}
\beta = \frac{K u_N - K_1 K^2}{K_1 R}
\end{equation}
\begin{equation}\label{eq10}
b = K_2 (\frac{K^2}{R} + \beta)
\end{equation}

W równaniu \ref{eq10} pominięto kwestie znaku wynikające z użycia funkcji $\sgn$.

Wartości współczynników $K_1$, $K_2$ odczytane z wykresów:

\begin{table}[h]
    \centering
        \caption{Współczynniki regresji liniowej.}
        \label{tab:wsp_regr_lin}
        
        \begin{tabularx}{1\textwidth}{l | l | l}
            \toprule
            Parametr & Ujemna prędkość obrotowa & Dodatnia prędkość obrotowa \\
            \midrule
            $K_1$ & \num{5,0591527273} & \num{1,0680286292} \\
            $K_2$ & \num{5,068153533} & \num{-1,0789743494} \\
            \bottomrule
        \end{tabularx}
\end{table}

\end{document}