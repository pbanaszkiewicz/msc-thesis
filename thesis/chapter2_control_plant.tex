\chapter{Obiekt regulacji}
\label{cha:ch2_obiekt_regulacji}

Obiektem poddawanym regulacji był system typu kulka na belce, który został zbudowany od podstaw na potrzeby tej pracy.

%%%%%%%%
\section{Obiekty typu kulka na belce}

Na system tego typu składają się długa, umieszczona horyzontalnie belka, łożyskowana w sposób umożliwiający zmianę kąta nachylenia, i silnik lub serwomechanizm, który umożliwia wychylanie belki.
Po belce swobodnie toczy się kulka.

Podstawowym zadaniem regulacji w systemie tego typu jest stabilizacja położenia kulki w wybranym punkcie.
Charakterystyczną cechą tego systemu jest prostota konstrukcji oraz niestabilność przy braku aktywnej regulacji.

Obiekty tego typu są często wykorzystywane w dydaktyce teorii sterowania. Składają się na to poniższe powody:

\begin{itemize}
	\item prostota budowy,
	\item możliwość zastosowania różnych czujników położenia kulki,
	\item możliwość zastosowania różnych silników i mechanizmów przeniesienia napędu,
    \item możliwość zastosowania różnych struktur i algorytmów sterowania,
    \item stosunkowo wysoki (czwarty) rząd modelu, jego nieliniowość i niestabilność.
\end{itemize}

Uproszczony schemat systemu kulka i belka przedstawiony został na rysunku \ref{fig:schemat_uproszczony}:

\begin{figure}[H]
	\centering
    \includesvg[width=0.7\textwidth,svgpath=./vector_graphics/]{schemat_uproszczony}
	\caption{Uproszczony schemat systemu typu kulka i belka.}
	\label{fig:schemat_uproszczony}
\end{figure}

Prostota konstrukcji i inherentna niestabilność sprawiły, że powstało wiele implementacji tego systemu (np. \cite{BABEX1}\cite{BABEX2}\cite{BABEX3}), również komercyjne, jak na przykład produkt firmy Quanser (\cref{fig:quanser_ball_beam}):

\begin{figure}[H]
	\centering
	\includegraphics[width=0.5\textwidth]{quanser_ball_beam}
	\caption{Zdjęcie produktu \textit{Ball and Beam} firmy Quanser. Źródło: \url{http://www.quanser.com/Products/ball_beam}.}
	\label{fig:quanser_ball_beam}
\end{figure}

%%%%%%%%
\section{Projekt mechaniczny}

Przed przystąpieniem do budowy obiektu, zaprojektowano wstępny kształt w programie  \textsc{SketchUp Make} (\cref{fig:cad_render}). Wyszczególniono na nim:

\begin{itemize}
	\item prostokątną podstawę,
	\item słupy podtrzymujące belkę,
	\item usztywniający łącznik między słupami,
	\item oś obrotu (wał) umieszczony w połowie długości belki,
	\item przekrój belki.
\end{itemize}

\begin{figure}[H]
	\centering
	\includegraphics[width=0.32\textwidth]{cad}
	\caption{Render projektu CAD.}
	\label{fig:cad_render}
\end{figure}

Ostateczna konstrukcja różni się od projektu \textsc{CAD} o wysokość słupów i umiejscowienie łącznika między nimi. Dodatkowo zastosowano sztywne połączenie osi obrotu belki i samej belki wykorzystujące łożyskowane podpory wału.

%%%%%%%%
\section{Konstrukcja mechaniczna}
\label{sec:ch2_konstrukcja_mechaniczna}

Większość elementów konstrukcji (\cref{fig:perspektywa}) powstało z ocynkowanych segmentów stalowych, tzw. ceowników w przekroju kwadratowym o boku długości \SI{4}{cm}, pozwalających na łatwe łączenie kilku elementów przy pomocy śrub. Rozwiązanie to jest bardzo tanie w porównaniu do przemysłowych profili aluminiowych lub spawanych profili stalowych, ale jednocześnie jest dość ciężkie i poprzez niedomknięcie profilu podatne na pewne momenty gnące.

Kąty proste pomiędzy elementami ustawionymi prostopadle zostały zapewnione poprzez zastosowanie kątowych wsporników stalowych.

\begin{figure}[H]
	\centering
	\includegraphics[width=0.9\textwidth]{perspektywa2}
	\caption{Zdjęcie obiektu regulacji z zaznaczonymi poszczególnymi elementami konstrukcji; w środku obiektu opakowanie zapałek dla porównania rozmiarów.}
	\label{fig:perspektywa}
\end{figure}

\begin{figure}[p]
    \centering
    \includegraphics[width=0.5\textwidth]{podpory_belki_lozyska}
    \caption{Zdjęcie mocowania belki do wału przy użyciu podpór; widoczne łożyskowanie wału.}
    \label{fig:podpory_belki_lozyska}
\end{figure}

\begin{figure}[p]
    \centering
    \includegraphics[width=0.3\textwidth]{naped_bazowanie}
    \caption{Zdjęcie układu przeniesienia napędu z widocznym korbowodem, korbą, silnikiem, chorągiewką używaną do bazowania i układem bazowania.}
    \label{fig:naped_bazowanie}
\end{figure}

Na prostokątnej podstawie o wymiarach zewnętrznych \SI{60 x 23}{cm} wykonanej z ceowników ustawiono pionowo na środkach dłuższych boków słupy nośne, również wykonane z ceowników. Słupy zostały usztywnione poprzez połączenie ich przęsłem podniesionym o \SI{11}{cm} względem podstawy.

Na słupach przyczepiono współosiowo łożyska maszynowe samonastawne typu UCFL 201 w obudowach odlewanych. Przez łożyska poprowadzono pręt nierdzewny stalowy o średnicy \SI{12}{mm}; na pręt nałożono podpory wałka w kształcie litery \texttt{T}, a do nich przykręcono belkę. Mocowanie belki zostało przedstawione na \cref{fig:podpory_belki_lozyska}.

Silnik elektryczny, przymocowany do aluminiowego uchwytu, został umieszczony podłużnie na krótszym boku podstawy, na podwyższeniu wykonanym z dwóch elementów stalowych typu ceownik.

%%%%%%%%
\section{Przeniesienie napędu}
\label{sec:ch2_przeniesienie_napedu}

W obiekcie zastosowano przeniesienie napędu wykorzystujące mechanizm korbowy. Rozwiązanie to posiada kilka zalet:

\begin{itemize}
	\item gwarantuje bezpieczeństwo mechanizmu --- błąd algorytmiczny (np. przypadkowe podanie przez dłuższy czas maksymalnego sterowania) nie spowoduje uszkodzenia fizycznego żadnej części obiektu,
	\item poprzez oddalenie punktu zaczepu korbowodu od osi obrotu belki zmniejsza wymagania dotyczące momentu obrotowego rozwijanego przez silnik, a tym samym jego cenę,
	\item pozwala regulować zakres wychyleń belki w wyniku zmiany długości korby.
\end{itemize}

\begin{figure}[H]
	\centering
	\includesvg[width=0.6\textwidth,svgpath=./vector_graphics/]{mechanizm_korbowy}
	\caption{Schemat napędu opartego o mechanizm korbowy.}
	\label{fig:mechanizm_korbowy}
\end{figure}

Parametry fizyczne mechanizmu korbowego:

\begin{itemize}
	\item długość korby: \SI{3}{cm},
	\item długość korbowodu: \SI{16}{cm},
	\item zastosowane przeguby kulowe między korbą i korbowodem oraz korbowodem i belką,
    \item użyty silnik prądu stałego, komutatorowy, z magnesami trwałymi, sprzężony z zębatą przekładnią redukcyjną (więcej w rozdziale \ref{sec:ch3_uklad_napedowy}).
\end{itemize}

Alternatywne układy przeniesienia napędu i mechanizmy elektromotoryczne zostały przedyskutowane w dodatku \ref{appA_warianty_zespolu_napedowego}.

Mechanizm przeniesienia napędu wraz z widocznym silnikiem i układem bazowania został zaprezentowany na \cref{fig:naped_bazowanie}.

%%%%%%%%

\section{Belka}
\label{sec:ch2_belka}

Belka została stworzona poprzez trwałe sklejenie krawędzi kątownika aluminiowego o długości \SI{40}{cm} i boku \SI{3}{cm} oraz krawędzi ceownika aluminiowego o długości \SI{65}{cm} i boku \SI{4}{cm}. W przekroju przypomina to kształtem literę \texttt{M} domkniętą od spodu (\cref{fig:przekroj_belki}).

\begin{figure}[h]
	\centering
	\includesvg[width=0.2\textwidth,svgpath=./vector_graphics/]{przekroj_belki}
    % \includegraphics[width=0.2\textwidth]{beam_xsection}
	\caption{Schemat przekroju belki z zaznaczonymi ceownikiem aluminiowym A) i~kątownikiem aluminiowym B).}
	\label{fig:przekroj_belki}
\end{figure}

Użyty kątownik jest nieco krótszy od ceownika. Zamocowano go symetrycznie, a w odległościach około \SI{1}{cm} od jego końców zamontowano wsporniki (\cref{fig:uchwyt_czujnika_odleglosci}) na czujniki optyczne (zob. rozdział \ref{sec:ch3_czujniki_odleglosci}).

Wsporniki pozwalają na zmianę wysokości czujnika względem płaszczyzny belki, a także na pochylenie go w osi prostopadłej do płaszczyzny belki.

\begin{figure}[h]
    \centering
    \includesvg[width=0.32\textwidth,svgpath=./vector_graphics/]{wspornik_czujnika}
    \caption{Schemat uchwytu na czujnik odległości w rzucie z przodu i z boku. Zastosowanie mocowania na śrubie pozwala pochylać czujnik względem belki.}
    \label{fig:uchwyt_czujnika_odleglosci}
\end{figure}

%%%%%%%%
\section{Kulka}
\label{sec:ch2_kulka}

Do projektu dobrano lekką kulkę (\cref{fig:kulka}) o masie \SI{20}{g} wykonaną z miękkiej pianki; średnica kulki wynosi \SI{6}{cm}.

\begin{figure}[h]
    \centering
    \includegraphics[width=0.3\textwidth]{ball}
    \caption{Zdjęcie kulki.}
    \label{fig:kulka}
\end{figure}

%%%%%%%%
\section{Podsumowanie}

W niniejszym rozdziale przedstawiono obiekt regulacji, cel regulacji oraz przedstawiono podobne konstrukcje, w tym jedno rozwiązanie komercyjne. Następnie opisano dokładnie budowę obiektu regulacji, poczynając od konstrukcji podstawy, poprzez umocowanie osi obrotu belki, umieszczenie silnika, przeniesienie napędu, a na budowie belki i doborze kulki kończąc.

%---------------------------------------------------------------------------