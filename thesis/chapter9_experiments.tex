\chapter{Symulacje i eksperymenty}
\label{cha:ch9_symulacje_i_eksperymenty}

Aby sprawdzić poprawność działania modelów nieliniowego i liniowego (rozdział \ref{sec:ch6_punkt_pracy_linearyzacja}) obiektu przeprowadzono szereg opisanych w tym rozdziale symulacji. Sprawdzono również działanie regulatorów zaimplementowanych w rzeczywistym obiekcie regulacji, a także porównano wynik działania regulatorów dobranych matematycznie (rozdziały \ref{sec:ch6_regulator_belki} oraz \ref{sec:ch6_regulator_kulki}) z regulatorem dostrojonym w procedurze samostrojenia (rozdział \ref{cha:ch8_algorytmy_samostrojenia}).

%%%%
\section{Symulacje modelów}
\label{sec:ch9_symulacje_modelow}

%W celu sprawdzenia działania modelów nieliniowego i liniowego zarejestrowano ich odpowiedzi przy braku wymuszenia. Dla modelu nieliniowego stan początkowy to $\begin{bmatrix}
%0 & 0 & 0 & 0
%\end{bmatrix}$, natomiast dla liniowego stan początkowy zmieniono odrobinę tak, aby wyprowadzić go z równowagi: $\begin{bmatrix}
%0,001 & 0 & 0,001 & 0
%\end{bmatrix}$.

W celu sprawdzenia działania modelów nieliniowego i liniowego zarejestrowano ich odpowiedzi przy braku aktywnej regulacji. W przypadku obu modeli stanem początkowym były wartości $\begin{bmatrix}
0 & 0 & 0 & 0
\end{bmatrix}$, natomiast model liniowy, z racji tego, że jest oparty o odchyłki od wybranego punktu równowagi, w trakcie symulacji miał podawaną stałą wartość $-u^*=0,51158$ na wejście.

\begin{sidewaysfigure}
    \centering
    \begin{subfigure}[b]{0.49\textwidth}
        \includesvg[width=1\textwidth,svgpath=./vector_graphics/,pretex=\relscale{0.6}]{model_nieliniowy_odpowiedz}
        \caption{Odpowiedź modelu nieliniowego przy braku sterowania.}
        \label{fig:odpowiedz_modelu_nieliniowego_brak_sterowania}
    \end{subfigure}
    \begin{subfigure}[b]{0.49\textwidth}
        \includesvg[width=1\textwidth,svgpath=./vector_graphics/,pretex=\relscale{0.6}]{model_liniowy_odpowiedz}
        \caption{Odpowiedź modelu liniowego przy braku sterowania.}
        \label{fig:odpowiedz_modelu_liniowego_brak_sterowania}
    \end{subfigure}
    
    \caption{Odpowiedzi modeli obiektu regulacji typu kulka i belka.}\label{fig:odpowiedzi_modeli}
\end{sidewaysfigure}

Wykresy odpowiedzi modeli dla każdej zmiennej stanu zostały przedstawione na \cref{fig:odpowiedzi_modeli}. Należy zauważyć, że model nieliniowy został zbudowany w taki sposób, że belka tworzy nieograniczoną powierzchnię staczania dla kulki --- w momencie, gdy fizyczna kulka w rzeczywistym obiekcie regulacji uderza o ogranicznik, kulka w modelu nieliniowym stacza się dalej wzdłuż niewidocznego przedłużenia belki. 
Z tego powodu zdecydowano się ograniczyć czas symulacji do \SI{1}{\second}, gdyż to wtedy kulka osiąga pozycję bliską końca belki.

Kąt obrotu belki w modelu nieliniowym nie rośnie stale, tak jak to ma miejsce w przypadku modelu liniowego, ale można zauważyć, że pozycje i prędkości kątowe na obu modelach przechodzą praktycznie przez te same punkty dla czasu \SI{0.2}{\second}.

Odpowiedź modelu liniowego, w przypadku położenia i prędkości liniowej kulki, dobrze zachowuje charakter odpowiedzi dla analogicznych zmiennych modelu nieliniowego. Przyspieszenie liniowe, którego doznaje kulka, jest większe w przypadku modelu liniowego, co wiąże się ze stale rosnącym kątem pochylenia belki, a --- jak wykazano w rozdziale \ref{sec:ch8_identyfikacja_kulki} --- to od niego zależy liniowo przyspieszenie kulki; ostatecznie kulka w modelu liniowym osiąga wyższą prędkość niż w modelu nieliniowym.

Zatem zachowanie modelu zlinearyzowanego w otoczeniu punktu równowagi (zerowe położenie i~prędkość kulki oraz belki) odpowiada oczekiwaniom; ponadto jest ono jedynie przybliżeniem działania modelu nieliniowego, przez co w trakcie oddalania od punktu równowagi zachowanie obu modeli rozbiega się coraz bardziej.

%%%%
\section{Symulacje działania regulatorów}
\label{sec:ch9_symulacje_regulatorow}

Kolejnym porównaniem, któremu poddane zostały model nieliniowy oraz liniowy obiektu regulacji, była stabilizacja kulki w zerowym położeniu równowagi. Wartościami początkowymi zmiennych stanu były: \SI{-0.1}{\meter} dla pozycji kulki, oraz \num{0} dla jej prędkości oraz pochylenia i prędkości kątowej belki. W~symulacjach wykorzystano regulatory w konfiguracji kaskadowej (opisane w rozdziale \ref{sec:ch6_kaskadowy_uklad_regulacji}), przy czym w~modelu nieliniowym szeregowo ułożone regulatory oddziaływały na silnik, a stany odczytywane były z~odpowiednich złącz (zob. rozdział \ref{cha:ch4_model_symulacyjny}), natomiast schemat regulacji wykorzystujący model liniowy został zbudowany jak na \cref{fig:schemat_regulacji_belka_i_kulka}.

\begin{figure}[h]
    \includesvg[width=1\textwidth,svgpath=./vector_graphics/,pretex=\relscale{0.6}]{model_nieliniowy_stabilizacja}
    \caption{Stabilizacja położenia kulki w modelu nieliniowym.}
    \label{fig:stabilizacja_modelu_nieliniowego}
\end{figure}

\begin{figure}[h]
    \includesvg[width=1\textwidth,svgpath=./vector_graphics/,pretex=\relscale{0.6}]{model_liniowy_stabilizacja}
    \caption{Stabilizacja położenia kulki w modelu liniowym.}
    \label{fig:stabilizacja_modelu_liniowego}
\end{figure}

Wykresy przedstawiające położenie liniowe kulki (część lewa), pozostałe zmienne stanu (część środkowa) oraz sterowanie (z obu regulatorów) przedstawiono na \cref{fig:stabilizacja_modelu_nieliniowego} oraz \cref{fig:stabilizacja_modelu_liniowego}. Na wykresach obu regulatorów widać oscylacje pozycji kulki, których należało się spodziewać, gdyż dwie wartości własne macierzy ,,zwiniętego'' układu zlinearyzowanego wraz z regulatorami \eqref{eq:wartosci_wlasne_A_c} są zespolone.

%%%%
\section{Eksperymenty przeprowadzone na obiekcie rzeczywistym}
\label{sec:ch9_eksperymenty}

Wyniki eksperymentów przeprowadzonych na obiekcie rzeczywistym zostały zebrane, podobnie jak charakterystyki czujników odległości czy odpowiedź silnika elektrycznego, za pomocą narzędzia \texttt{Traces} z programu \textsc{Siemens TIA Portal}. Wydajność tego narzędzia jest ograniczona przez liczbę zbieranych sygnałów i pamięć udostępnioną przez sterownik dla tego narzędzia \cite{SIEMENSTRACE}, a w związku z~przeprowadzeniem rejestracji sygnałów każdej ze zmiennych stanu oraz rejestru binarnego informującego o~obecności kulki ilość próbek jest niewielka, zdecydowano się na zwiększenie okresu próbkowania narzędzia na około \SI{5}{\milli\second}\footnote{Próbkowanie może odbywać się w jednym z wielu dostępnych bloków organizacyjnych \texttt{OB}. Wszystkie pomiary (poza charakterystykami czujników odległości) wykorzystujące narzędzie \texttt{Traces} zostały ustawione na próbkowanie razem z \texttt{OB1}, a skoro czas cyklu procesora wynosił przeciętnie \SI{1}{\milli\second}, to i okres próbkowania \texttt{Traces} był odpowiednio $N$ razy większy.}.

%%%%
\subsection{Stabilizacja położenia różnych kulek}
\label{subsec:ch9_stabilizacja_polozenia_roznych_kulek}

Do eksperymentu porównującego stabilizację położenia różnych kulek wybrano, poza kulką opisaną w rozdziale \ref{sec:ch2_kulka}, podobną kulkę wykonaną również z pianki, ale o masie \SI{30}{\gram} i średnicy \SI{9}{\centi\meter}. Każda z kulek została ułożona na pozycji \SI{-12}{\centi\meter} od środka belki w momencie gdy belka była w~poziomie. Regulacja rozpoczynała się od razu po wykryciu przez sterownik obecności kulki.

Wykresy zarejestrowanych zmiennych stanu podczas eksperymentów stabilizacji położenia kulek przedstawiono na \cref{fig:stabilizacja_kulka1} i \cref{fig:stabilizacja_kulka2}; w lewej części przedstawiona jest pozycja kulki, natomiast pozostałe zmienne stanu w prawej części. Jak widać czas regulacji w przypadku mniejszej kulki, do której został dostrojony matematycznie model obiektu (zob. rozdział \ref{cha:ch6_model_liniowy}), wynosi około \SI{2}{\second}, a liczba przeregulowań to~2.

Wykres pozycji drugiej kulki (\cref{fig:stabilizacja_kulka2}) ukazuje większe oscylacje wokół zadanego punktu równowagi; czas regulacji również jest wydłużony i wynosi około \SI{9}{\second}. Oznacza to, że wybrany regulator gorzej radzi sobie z większą i cięższą piłką.

Dodatkowo na wykresie na \cref{fig:stabilizacja_kulka1} widoczny jest uchyb statyczny, ale nie jest on widoczny na wykresie dla drugiej kulki. Wynika to z faktu, że pierwsza kulka mogła zatrzymać się z powodu tarcia statycznego.

\begin{figure}[p]
    \includesvg[width=1\textwidth,svgpath=./vector_graphics/,pretex=\relscale{0.7}]{ball1_stabilization}
    \caption{Stabilizacja położenia pierwszej kulki opisanej w rozdziale \ref{sec:ch2_kulka}.}
    \label{fig:stabilizacja_kulka1}
\end{figure}
\begin{figure}[p]
    \includesvg[width=1\textwidth,svgpath=./vector_graphics/,pretex=\relscale{0.7}]{ball2_stabilization}
    \caption{Stabilizacja położenia drugiej, dodatkowej kulki.}
    \label{fig:stabilizacja_kulka2}
\end{figure}

%%%%
\subsection{Odpowiedź regulatora belki przed i po samostrojeniu}
\label{subsec:ch9_odp_regulatora_belki}

Odpowiedź regulatora belki została wyznaczona w eksperymencie, który polegał na dojechaniu do pozycji zadanej \SI{-0.2}{\radian} z pozycji bazowej (około \SI{0.175}{\radian}). Jak widać na wykresach \cref{fig:stabilizacja_belki_oryg} oraz \cref{fig:stabilizacja_belki_nastrojone}, regulator oryginalny osiągnął dokładny punkt zadany, ale gdyby nie tarcie statyczne przekładni, łożysk i przegubów, prawdopodobnie nastąpiłoby przeregulowanie, co widać po napięciu sterowania. Niemniej jednak czas regulacji tego regulatora to \SI{1}{\second}.

Regulator z nastawami dobranymi w procesie samostrojenia $K_I=\begin{bmatrix}
8,95713 & 1,071005
\end{bmatrix}$ również osiągnął (a nawet przekroczył) zadany punkt; dodatkowo na wykresie napięcia sterowania widać, że nastąpiłoby przeregulowanie, gdyby nie obecność tarcia statycznego w przekładni, łożyskach i przegubach. W przypadku tego regulatora czas regulacji jest jednak niższy (około \SI{0.7}{\second} licząc od rozpoczęcia regulacji do ustalenia wyjścia).

Jednocześnie można zaobserwować, że napięcie sterowania w przypadku obydwóch regulatorów nie osiąga nawet połowy wartości znamionowej. Z jednej strony oznacza to, że dynamika regulatorów jest mocno ograniczona, a z drugiej strony oznacza, że nie powinna nastąpić sytuacja podrzucenia kulki lub jej spadku swobodnego w wyniku szybkiego ruchu belką.

\begin{figure}[p]
    \includesvg[width=1\textwidth,svgpath=./vector_graphics/,pretex=\relscale{0.7}]{beam_untuned}
    \caption{Regulacja pozycji belki na oryginalnych nastawach.}
    \label{fig:stabilizacja_belki_oryg}
\end{figure}
\begin{figure}[p]
    \includesvg[width=1\textwidth,svgpath=./vector_graphics/,pretex=\relscale{0.7}]{beam_tuned}
    \caption{Regulacja pozycji belki na nowych, automatycznie dobranych nastawach.}
    \label{fig:stabilizacja_belki_nastrojone}
\end{figure}

%%%%
\subsection{Stabilizacja położenia różnych kulek po samostrojeniu regulatora belki}
\label{subsec:ch9_stabilizacja_polozenia_roznych_kulek_po_samostrojeniu}

W tym eksperymencie porównano jakość regulacji dla takich samych kulek, jak w rozdziale \ref{subsec:ch9_stabilizacja_polozenia_roznych_kulek}, przy użyciu kryteriów czasowych (przeregulowania, czas regulacji). Różnicą w porównaniu z eksperymentem z rozdziału \ref{subsec:ch9_stabilizacja_polozenia_roznych_kulek} jest uprzednie przeprowadzenie samostrojenia regulatora belki według algorytmu opisanego w rozdziale \ref{sec:ch8_algorytm_identyfikacji_belki}. W wyniku tego procesu otrzymano następujące nastawy regulatora belki: $K_I = \begin{bmatrix}
8,434616 & 1,022948
\end{bmatrix}$\footnote{Nastawy zostały uzyskane w przeprowadzonej drugi raz tej samej procedurze samostrojenia, dlatego ich wartość różni się od wartości $K_I$ umieszczonej w rozdziale \ref{subsec:ch9_odp_regulatora_belki}.}.

W przypadku mniejszej z kulek (wykres na \cref{fig:stabilizacja_kulka1_nastrojona}) czas regulacji uległ skróceniu z \SI{2}{\second} do około \SI{1.8}{\second}, przy czym przeregulowanie jest jedno i widoczny jest uchyb ustalony. Może być on spowodowany tarciem statycznym pomiędzy kulką i jej podłożem.

Dużej poprawie uległa regulacja drugiej, większej i cięższej kulki, zaprezentowana na \cref{fig:stabilizacja_kulka2_nastrojona}. Widoczne jest tylko jedno przeregulowanie, a czas regulacji skrócił się z \SI{9}{\second} do około \SI{3}{\second}. Pojawiły się natomiast delikatne oscylacje oraz uchyb. Na wykresie po prawej stronie widoczne są zmiany kąta belki, co oznacza, że regulator nadrzędny zbyt mocno reaguje na zmiany jednej ze zmiennych stanu.

\begin{figure}[p]
    \includesvg[width=1\textwidth,svgpath=./vector_graphics/,pretex=\relscale{0.7}]{ball1_tuned_stabilization}
    \caption{Stabilizacja położenia pierwszej kulki przy użyciu nastrojonego regulatora belki.}
    \label{fig:stabilizacja_kulka1_nastrojona}
\end{figure}
\begin{figure}[p]
    \includesvg[width=1\textwidth,svgpath=./vector_graphics/,pretex=\relscale{0.7}]{ball2_tuned_stabilization}
    \caption{Stabilizacja położenia drugiej kulki przy użyciu nastrojonego regulatora belki.}
    \label{fig:stabilizacja_kulka2_nastrojona}
\end{figure}

%%%%
\section{Podsumowanie}

W niniejszym rozdziale przedstawiono kilka symulacji i eksperymentów pokazujących zachowanie się modeli oraz regulatorów w różnych sytuacjach.

Na początku zaprezentowano i potwierdzono zgodność modelu liniowego obiektu z modelem nieliniowym, następnie zaprezentowano działanie wyliczonych matematycznie (rozdział \ref{cha:ch6_model_liniowy}) regulatorów z~każdym z~tych modeli.

W następnej części rozdziału zaprezentowano wyniki eksperymentów przeprowadzonych na obiekcie rzeczywistym z zaimplementowanymi regulatorami i mechanizmem samostrojenia regulatora podrzędnego. W wynikach znajduje się porównanie jakości regulacji przy użyciu kryteriów czasowych dla następujących eksperymentów:
\begin{itemize}
    \item porównanie działania regulatorów wyznaczonych matematycznie dla modelu przy zastosowaniu oryginalnej kulki oraz innej, większej i cięższej kulki,
    \item porównanie regulacji regulatora podrzędnego przed i po procedurze samostrojenia,
    \item ponowne porównanie działania obiektu regulacji dla dwóch kulek, ale po przeprowadzonej procedurze samostrojenia regulatora podrzędnego.
\end{itemize}

Po przeprowadzeniu procedury samostrojenia wyniki regulacji dla kulki, która nie brała udziału w~matematycznym wyznaczaniu regulatorów, okazały się dużo lepsze niż przed samostrojeniem.

%%%%
% pomysły na eksperymenty:
% - nastrojony przez autotune'a w TIA Portal regulator PID - umiejętność stabilizacji kulki piankowej
% - jw. zmiana kulki na piłkę tenisową bez przeprowadzania autotune'a
% - jw. ale po przeprowadzeniu autotune'a
% - regulatory od stanu obliczone analitycznie (kulka piankowa)
% - regulatory od stanu obliczone analitycznie (piłka tenisowa)
% - samostrojenie z kulką piankową i rezultaty
% - jw. z piłką tenisową i rezultaty
% - przestrojenie na piłkę tenisową i rezultaty

% do tego symulacje:
% - porównanie zachowania modelu nieliniowego i liniowego
% - porównanie zachowania modelu liniowego i rzeczywistego
% - porównanie zachowania modelu nieliniowego i rzeczywistego

%---------------------------------------------------------------------------