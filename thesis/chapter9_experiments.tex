\chapter{Symulacje i eksperymenty}
\label{cha:ch9_symulacje_i_eksperymenty}

Aby sprawdzić poprawność działania modelów nieliniowego i liniowego (rozdział \ref{sec:ch6_punkt_pracy_linearyzacja}) obiektu przeprowadzono szereg opisanych w tym rozdziale symulacji. Sprawdzono również działanie regulatorów zaimplementowanych w rzeczywistym obiekcie regulacji, a także porównano wynik działania regulatorów dobranych matematycznie (rozdziały \ref{sec:ch6_regulator_belki} oraz \ref{sec:ch6_regulator_kulki}) z regulatorem dostrojonym w procedurze samostrojenia (rozdział \ref{cha:ch8_algorytmy_samostrojenia}).

%%%%
\section{Symulacje modelów}
\label{sec:ch9_symulacje_modelow}

W celu sprawdzenia działania modelów nieliniowego i liniowego zarejestrowano ich odpowiedzi przy braku wymuszenia. Dla modelu nieliniowego stan początkowy to $\begin{bmatrix}
0 & 0 & 0 & 0
\end{bmatrix}$, natomiast dla liniowego stan początkowy zmieniono odrobinę tak, aby wyprowadzić go z równowagi: $\begin{bmatrix}
0,001 & 0 & 0,001 & 0
\end{bmatrix}$.

\begin{sidewaysfigure}
    \centering
    \begin{subfigure}[b]{0.49\textwidth}
        \includesvg[width=1\textwidth,svgpath=./vector_graphics/,pretex=\relscale{0.7}]{model_nieliniowy_odpowiedz}
        \caption{Odpowiedź modelu nieliniowego przy braku sterowania.}
        \label{fig:odpowiedz_modelu_nieliniowego_brak_sterowania}
    \end{subfigure}
    \begin{subfigure}[b]{0.49\textwidth}
        \includesvg[width=1\textwidth,svgpath=./vector_graphics/,pretex=\relscale{0.7}]{model_liniowy_odpowiedz}
        \caption{Odpowiedź modelu liniowego przy braku sterowania.}
        \label{fig:odpowiedz_modelu_liniowego_brak_sterowania}
    \end{subfigure}
    
    \caption{Odpowiedzi modeli obiektu regulacji typu kulka i belka.}\label{fig:odpowiedzi_modeli}
\end{sidewaysfigure}

Wykresy odpowiedzi modeli dla każdej zmiennej stanu zostały przedstawione na \cref{fig:odpowiedzi_modeli}. Należy zauważyć, że model nieliniowy został zbudowany w taki sposób, że belka tworzy nieograniczoną powierzchnię staczania dla kulki -- w momencie, gdy fizyczna kulka w rzeczywistym obiekcie regulacji uderza o ogranicznik, kulka w modelu nieliniowym stacza się dalej wzdłuż niewidocznego przedłużenia belki. Widoczne jest to po czasie \SI{1}{\second} na wykresie \ref{fig:odpowiedz_modelu_nieliniowego_brak_sterowania}. Dodatkowo w tej chwili widać, że belka osiąga swoje maksymalne położenie kątowe.

Odpowiedź modelu nieliniowego, w przypadku położenia i prędkości liniowej kulki, wiernie odwzorowuje odpowiedź dla analogicznych zmiennych modelu nieliniowego, przy czym przyspieszenie, jakiemu poddana jest kulka, jest mniejsze. Oznacza to dobrą realizację przybliżenia liniowego w otoczeniu punktu równowagi kulki.

Mniejsze przyspieszenie kulki w modelu liniowym wynika z małego kąta, który osiąga belka. Zachowanie belki w pobliżu punktu równowagi jest zbliżone do modelu nieliniowego, natomiast poza nim na belkę działa dużo mniejsze przyspieszenie kątowe niż w modelu nieliniowym.

Zatem zachowanie modelu zlinearyzowanego w otoczeniu punktu równowagi (zerowe położenie i~prędkość kulki oraz belki) odpowiada oczekiwaniom; ponadto jest ono faktycznym przybliżeniem działania modelu nieliniowego, przez co w trakcie oddalania od punktu równowagi zachowanie obu modeli rozbiega się coraz bardziej.

%%%%
\section{Symulacje działania regulatorów}
\label{sec:ch9_symulacje_regulatorow}

Kolejnym porównaniem, któremu poddane zostały model nieliniowy oraz liniowy obiektu regulacji, była stabilizacja kulki w zerowym położeniu równowagi. Wartościami początkowymi zmiennych stanu były: \SI{-0.1}{\meter} dla pozycji kulki, oraz \num{0} dla jej prędkości oraz pochylenia i prędkości kątowej belki. W symulacjach wykorzystano regulatory w konfiguracji kaskadowej (opisane w rozdziale \ref{sec:ch6_kaskadowy_uklad_regulacji}), przy czym w obiekcie nieliniowym szeregowo ułożone regulatory oddziaływały na silnik, a stany odczytywane były z odpowiednich złącz (zob. rozdział \ref{cha:ch4_model_symulacyjny}), natomiast schemat regulacji wykorzystujący model liniowy został zbudowany jak na \cref{fig:schemat_regulacji_belka_i_kulka}.

\begin{figure}[h]
    \includesvg[width=1\textwidth,svgpath=./vector_graphics/,pretex=\relscale{0.6}]{model_nieliniowy_stabilizacja}
    \caption{Stabilizacja położenia kulki w modelu nieliniowym.}
    \label{fig:stabilizacja_modelu_nieliniowego}
\end{figure}

\begin{figure}[h]
    \includesvg[width=1\textwidth,svgpath=./vector_graphics/,pretex=\relscale{0.6}]{model_liniowy_stabilizacja}
    \caption{Stabilizacja położenia kulki w modelu liniowym.}
    \label{fig:stabilizacja_modelu_liniowego}
\end{figure}

Wykresy przedstawiające położenie liniowe kulki (część lewa), pozostałe zmienne stanu (część środkowa) oraz sterowanie (z obu regulatorów) przedstawiono na \cref{fig:stabilizacja_modelu_nieliniowego} oraz \cref{fig:stabilizacja_modelu_liniowego}. Można zaobserwować, że w przypadku modelu nieliniowego pojawia się uchyb ustalony pozycji (trochę ponad \SI{3}{\centi\meter}), co sugeruje, że w obiekcie rzeczywistym należałoby zastosować regulator z częścią całkującą. Na wykresie dla modelu liniowego nie ma uchybu ustalonego, za to widać pewne oscylacje o amplitudzie około \SI{3}{\milli\metre}, które w rzeczywistym układzie nie powinny być widoczne za sprawą obecności tarcia statycznego pomiędzy kulką i belką. Należało się spodziewać oscylacji bardzo wolno gasnących, gdyż dwie wartości własne macierzy ,,zwiniętego'' układu zlinearyzowanego wraz z regulatorami \eqref{eq:wartosci_wlasne_A_c} są zespolone, a ich mała część rzeczywista (\num{-0.0527}) świadczy o bardzo powolnym gaśnięciu oscylacji.

%%%%
\section{Eksperymenty przeprowadzone na obiekcie rzeczywistym}
\label{sec:ch9_eksperymenty}

Wyniki eksperymentów przeprowadzonych na obiekcie rzeczywistym zostały zebrane, podobnie jak charakterystyki czujników odległości czy odpowiedź silnika elektrycznego, za pomocą narzędzia \texttt{Traces} z programu \textsc{Siemens TIA Portal}.
% TODO: wodolejstwo o sposobie przeprowadzenia

%%%%
\subsection{Stabilizacja położenia różnych kulek}
\label{subsec:ch9_stabilizacja_polozenia_roznych_kulek}

Do eksperymentu porównującego stabilizację położenia różnych kulek wybrano, poza kulką opisaną w rozdziale \ref{sec:ch2_kulka}, podobną kulkę wykonaną również z pianki, ale o masie \SI{30}{\gram} i średnicy niecałych \SI{9}{\centi\meter}.

% TODO: wykres jak to zadziałało w przypadku różnych kulek

%%%%
\subsection{Odpowiedź regulatora belki przed i po samostrojeniu}
\label{subsec:ch9_odp_regulatora_belki}

% TODO: sama reakcja na dojazd z takiej samej pozycji do zadanej
% to wymaga zmian w sekwencji głównej

%%%%
\subsection{Stabilizacja położenia różnych kulek po samostrojeniu regulatora belki}
\label{subsec:ch9_stabilizacja_polozenia_roznych_kulek_po_samostrojeniu}

% TODO: wykres jak to zadziałało w przypadku różnych kulek po zmianie regulatora belki

%%%%
\section{Podsumowanie}

% TODO: dokończ

%%%%
% pomysły na eksperymenty:
% - nastrojony przez autotune'a w TIA Portal regulator PID - umiejętność stabilizacji kulki piankowej
% - jw. zmiana kulki na piłkę tenisową bez przeprowadzania autotune'a
% - jw. ale po przeprowadzeniu autotune'a
% - regulatory od stanu obliczone analitycznie (kulka piankowa)
% - regulatory od stanu obliczone analitycznie (piłka tenisowa)
% - samostrojenie z kulką piankową i rezultaty
% - jw. z piłką tenisową i rezultaty
% - przestrojenie na piłkę tenisową i rezultaty

% do tego symulacje:
% - porównanie zachowania modelu nieliniowego i liniowego
% - porównanie zachowania modelu liniowego i rzeczywistego
% - porównanie zachowania modelu nieliniowego i rzeczywistego

%---------------------------------------------------------------------------