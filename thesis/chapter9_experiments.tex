\chapter{Symulacje i eksperymenty}
\label{cha:ch9_symulacje_i_eksperymenty}

Aby sprawdzić poprawność działania modelów nieliniowego i liniowego (rozdział \ref{sec:ch6_punkt_pracy_linearyzacja}) obiektu przeprowadzono szereg opisanych w tym rozdziale symulacji. Sprawdzono również działanie regulatorów zaimplementowanych w rzeczywistym obiekcie regulacji, a także porównano wynik działania regulatorów dobranych matematycznie (rozdziały \ref{sec:ch6_regulator_belki} oraz \ref{sec:ch6_regulator_kulki}) z regulatorem dostrojonym w procedurze samostrojenia (rozdział \ref{cha:ch8_algorytmy_samostrojenia}).

%%%%
\section{Symulacje modelów}
\label{sec:ch9_symulacje_modelow}

% TODO: symulacja modelu nieliniowego i liniowego no chyba że nie wyjdzie

%%%%
\section{Symulacje działania regulatorów}
\label{sec:ch9_symulacje_regulatorow}

% TODO: symulacje regulatorów na modelu nieliniowym i liniowym

%%%%
\section{Eksperymenty przeprowadzone na obiekcie rzeczywistym}
\label{sec:ch9_eksperymenty}

%%%%
\subsection{Stabilizacja położenia różnych kulek}
\label{subsec:ch9_stabilizacja_polozenia_roznych_kulek}

%%%%
\subsection{Odpowiedź regulatora belki przed i po samostrojeniu}
\label{subsec:ch9_odp_regulatora_belki}

%%%%
\subsection{Stabilizacja położenia różnych kulek po samostrojeniu regulatora belki}
\label{subsec:ch9_stabilizacja_polozenia_roznych_kulek_po_samostrojeniu}

%%%%
\section{Podsumowanie}

%%%%
% pomysły na eksperymenty:
% - nastrojony przez autotune'a w TIA Portal regulator PID - umiejętność stabilizacji kulki piankowej
% - jw. zmiana kulki na piłkę tenisową bez przeprowadzania autotune'a
% - jw. ale po przeprowadzeniu autotune'a
% - regulatory od stanu obliczone analitycznie (kulka piankowa)
% - regulatory od stanu obliczone analitycznie (piłka tenisowa)
% - samostrojenie z kulką piankową i rezultaty
% - jw. z piłką tenisową i rezultaty
% - przestrojenie na piłkę tenisową i rezultaty

% do tego symulacje:
% - porównanie zachowania modelu nieliniowego i liniowego
% - porównanie zachowania modelu liniowego i rzeczywistego
% - porównanie zachowania modelu nieliniowego i rzeczywistego

%---------------------------------------------------------------------------