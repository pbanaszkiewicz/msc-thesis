\chapter{Algorytmy sterowania}
\label{cha:ch7_algorytmy_sterowania}

Obsługa czujników odległości i bazowania, a także algorytmy sterowania zostały zaimplementowane na przemysłowym sterowniku PLC Siemens S7-1211C. Schemat blokowy głównej pętli programu został przedstawiony na \cref{fig:schemat_petla_glowna_PLC}, natomiast szczegółowy opis poszczególnych elementów i algorytmów zamieszczono poniżej.

\begin{figure}[ht]
    \centering
    
    \begin{tikzpicture}[auto, node distance=1cm,>=latex']
    \node [roundblock] (start) {\textsc{START}};
    \node [block, below=of start] (sensors) {Odczyt czujników};
    \node [block, below=of sensors] (main_sequence) {Główna sekwencja};
    \node [block, below=of main_sequence] (outputs) {Zapis do wyjść};
    
    \draw [->] (start) -- (sensors);
    \draw [->] (sensors) -- (main_sequence);
    \draw [->] (main_sequence) -- (outputs);
    \draw [->] (outputs.west) -- ++ (-1,0) |- (sensors.west);
    \end{tikzpicture}
    
    \caption{Schemat pętli głównej programu.}
    \label{fig:schemat_petla_glowna_PLC}
\end{figure}

W ramach bloku \textit{Odczyt czujników} następują poniższe operacje:
\begin{itemize}
    \item obsługa enkodera:
    \begin{itemize}
        \item przeliczanie wartości licznika enkodera na kąt obrotu wału motoreduktora,
        \item przeliczanie kąta obrotu wału motoreduktora na kąt obrotu belki, zgodnie z informacjami podanymi w rozdziale \ref{sec:ch4_zaleznosc_kata_silnika_i_kata_belki},
    \end{itemize}
    \item obsługa czujników odległości:
    \begin{itemize}
        \item obliczanie odległości dla każdego czujnika przy wykorzystaniu punktów charakterystyki i algorytmu aproksymacji liniowej,
        \item obliczanie pozycji kulki oraz jej średnicy,
        \item weryfikacja obecności kulki (szerzej opisana w rozdziale \ref{sec:ch7_wykrywanie_braku_kulki}),
    \end{itemize}
    \item generowanie impulsów od zboczy: narastającego dla przycisku \texttt{START} (NO) oraz opadającego dla przycisku \texttt{STOP} (NC),
    \item generowanie impulsu od opadającego zbocza czujnika bazowania.
\end{itemize}

W programie obsługiwane są tylko \num{3} wyjścia sterownika. Dwa z nich służą do sterowania silnikiem (sygnały \textit{PWM}, \textit{kierunek}), trzeci steruje diodą sygnalizacyjną. Sygnał \textit{PWM} kontrolowany jest przez sterownik jako tzw. wyjście pulsacyjne (\textit{PTO}); rolą programisty jest dostarczenie odpowiedniej wartości liczbowej do konkretnego słowa wyjściowego sterownika (tutaj \texttt{\%QW1000}). Częstotliwość wyjścia pulsacyjnego ustawiono na \SI{20}{\kilo\hertz}, natomiast kontrolę wypełnienia na setki, tzn. wpisanie wartości \num{85} do słowa wyjściowego sygnału PWM oznacza wypełnienie \SI{85}{\percent}.

Z racji możliwości sterowania obrotami silnika w dwie strony, regulator pozycji belki może zwrócić wartość z zakresu \numrange[range-phrase={ do }]{-100}{100}; stąd w bloku \textit{Zapis do wyjść} następuje konwersja tej wartości na jej moduł i zapis tego modułu do \texttt{\%QW1000}, natomiast sygnał \textit{kierunek} ustawiony jest na \num{1} dla ujemnej wartości sterowania.

% TODO: schemat sekwencji głównej

% TODO: opis sekwencji głównej

% TODO: wywoływane OB30, OB31

%%%%
\section{Wykrywanie braku kulki}
\label{sec:ch7_wykrywanie_braku_kulki}

Jak to zostało wspomniane w rozdziale \ref{sec:ch5_identyfikacja_charakterystyk_czujnikow}, poza zebraniem charakterystyk czujników odległości dla kolejnych położeń kulki, sprawdzono również jak reagują czujniki, gdy pomiędzy nimi nie znajduje się żadna przeszkoda.

Pomiary przedstawiono na \cref{fig:czujniki_charakterystyka_bez_przeszkody}. Wynika z nich, że oba czujniki przy braku przeszkody mocno szumią; jednocześnie widać, że średnie sygnałów to około \num{700} i niecałe \num{500}, a także widać sporadyczne skoki (zwłaszcza czujnika lewego) do wartości w zakresie charakterystyk (por. \cref{fig:aproksymacja_czujnika_lewego}).

Biorąc powyższe fakty pod uwagę, zdecydowano się na następujący algorytm wykrywania obecności kulki pomiędzy czujnikami:
\begin{itemize}
    \item jeżeli wartość z przetwornika ADC czujnika lewego wynosi mniej niż \num{800} oraz wartość z przetwornika ADC czujnika prawego wynosi mniej niż \num{600}, to zresetuj zmienną binarną \textit{kulka obecna},
    \item jeżeli wartość z przetwornika ADC czujnika lewego wynosi więcej niż \num{900} oraz wartość z przetwornika ADC czujnika prawego wynosi więcej niż \num{900}, to ustaw zmienną binarną \textit{kulka obecna}.
\end{itemize}

W wyniku zastosowania powyższych warunków, błędnie określona obecność kulki będzie tylko w sytuacji chwilowego wystąpienia skoków wartości na obu czujnikach; brak kulki zostanie wykryty natychmiast po zaniknięciu obu skoków wartości.

\begin{figure}[ht]
    \centering
    \includesvg[width=1\textwidth,svgpath=./vector_graphics/]{bez_przeszkody}    
    \caption{Odpowiedź czujników, gdy pomiędzy nimi nie ma żadnej przeszkody.}
    \label{fig:czujniki_charakterystyka_bez_przeszkody}
\end{figure}

%%%%
\section{Bazowanie}
\label{sec:ch7_bazowanie}

% TODO: opisz ideę
% TODO: schemat blokowy algorytmu
% TODO: wartość enkodera

%%%%
\section{Regulatory}
\label{sec:ch7_regulatory}

% TODO: ogólnie opisać sposób implementacji regulatorów w sterowniku

%%%%
\subsection{Regulator belki}
\label{subsec:ch7_regulator_belki}

% TODO: opisać implementację regulatora belki

%%%%
\subsection{Regulator kulki}
\label{subsec:ch7_regulator_kulki}

% TODO: opisać implementację regulatora kulki

%%%%
\section{Podsumowanie}

% TODO: dokończ
%---------------------------------------------------------------------------