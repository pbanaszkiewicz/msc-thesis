\documentclass[11pt]{aghdpl}
% \documentclass[en,11pt]{aghdpl}  % praca w języku angielskim

% Lista wszystkich języków stanowiących języki pozycji bibliograficznych użytych w pracy.
% (Zgodnie z zasadami tworzenia bibliografii każda pozycja powinna zostać utworzona zgodnie z zasadami języka, w którym dana publikacja została napisana.)
\usepackage[english,polish]{babel}

% Użyj polskiego łamania wyrazów (zamiast domyślnego angielskiego).
\usepackage{polski}

\usepackage[utf8]{inputenc}

% dodatkowe pakiety

\usepackage{mathtools}
\usepackage{amsfonts}
\usepackage{amsmath}
\usepackage{amsthm}

% --- < bibliografia > ---

\usepackage[
style=numeric,
sorting=none,
%
% Zastosuj styl wpisu bibliograficznego właściwy językowi publikacji.
language=autobib,
autolang=other,
% Zapisuj datę dostępu do strony WWW w formacie RRRR-MM-DD.
urldate=iso8601,
% Nie dodawaj numerów stron, na których występuje cytowanie.
backref=false,
% Podawaj ISBN.
isbn=true,
% Nie podawaj URL-i, o ile nie jest to konieczne.
url=false,
%
% Ustawienia związane z polskimi normami dla bibliografii.
maxbibnames=3,
backend=biber,
]{biblatex}

\usepackage{csquotes}
% Ponieważ `csquotes` nie posiada polskiego stylu, można skorzystać z mocno zbliżonego stylu chorwackiego.
\DeclareQuoteAlias{croatian}{polish}

\addbibresource{bibliography.bib}

% Nie wyświetlaj wybranych pól.
%\AtEveryBibitem{\clearfield{note}}


% ------------------------
% --- < listingi > ---

% Użyj czcionki kroju Courier.
\usepackage{courier}

\usepackage{listings}
\lstloadlanguages{TeX}

\lstset{
	literate={ą}{{\k{a}}}1
           {ć}{{\'c}}1
           {ę}{{\k{e}}}1
           {ó}{{\'o}}1
           {ń}{{\'n}}1
           {ł}{{\l{}}}1
           {ś}{{\'s}}1
           {ź}{{\'z}}1
           {ż}{{\.z}}1
           {Ą}{{\k{A}}}1
           {Ć}{{\'C}}1
           {Ę}{{\k{E}}}1
           {Ó}{{\'O}}1
           {Ń}{{\'N}}1
           {Ł}{{\L{}}}1
           {Ś}{{\'S}}1
           {Ź}{{\'Z}}1
           {Ż}{{\.Z}}1,
	basicstyle=\footnotesize\ttfamily,
}

% ------------------------

\AtBeginDocument{
	\renewcommand{\tablename}{Tabela}
	\renewcommand{\figurename}{Fig.}
}

% ------------------------
% --- < tabele > ---

\usepackage{array}
\usepackage{tabularx}
\usepackage{multirow}
\usepackage{booktabs}
\usepackage{makecell}
\usepackage[flushleft]{threeparttable}
\usepackage[hidelinks, colorlinks=false]{hyperref}
\usepackage{longtable}

% defines the X column to use m (\parbox[c]) instead of p (`parbox[t]`)
\newcolumntype{C}[1]{>{\hsize=#1\hsize\centering\arraybackslash}X}

% lepsze pozycjonowanie figur
\usepackage{float}

% liczby i ich symbole wg SI
\usepackage{siunitx}

% listingi kodu
\usepackage{listings}

%---------------------------------------------------------------------------

\author{Piotr Banaszkiewicz}
\shortauthor{P. Banaszkiewicz}

\titlePL{Dobór algorytmów regulacji oraz samostrojenia dla sterownika PLC współpracującego z nieliniowym obiektem mechatronicznym}
\titleEN{Synthesis of control and self tuning algorithms for a PLC controlling a nonlinear mechatronic ball and beam plant}


\shorttitlePL{Dobór algorytmów regulacji oraz samostrojenia dla sterownika PLC współpracującego z nieliniowym obiektem mechatronicznym}
\shorttitleEN{Synthesis of control and self tuning algorithms for a PLC controlling a nonlinear mechatronic ball and beam plant}

\thesistype{Praca dyplomowa magisterska}

\supervisor{dr inż. Andrzej Tutaj}

\degreeprogramme{Automatyka i Robotyka}

\date{2017}

\department{Katedra Automatyki i Inżynierii Biomedycznej}

\faculty{Wydział Elektrotechniki, Automatyki,\protect\\[-1mm] Informatyki i Inżynierii Biomedycznej}
%\faculty{Faculty of Electrical Engineering, Automatics, Computer Science and Biomedical Engineering}

\acknowledgements{Serdecznie dziękuję opiekunowi pracy, Panu Doktorowi Andrzejowi Tutajowi, za niesioną pomoc i~zawsze dobrą radę.}


\setlength{\cftsecnumwidth}{10mm}

%---------------------------------------------------------------------------
\setcounter{secnumdepth}{4}

\begin{document}

\titlepages

% Ponowne zdefiniowanie stylu `plain`, aby usunąć numer strony z pierwszej strony spisu treści i poszczególnych rozdziałów.
\fancypagestyle{plain}
{
	% Usuń nagłówek i stopkę
	\fancyhf{}
	% Usuń linie.
	\renewcommand{\headrulewidth}{0pt}
	\renewcommand{\footrulewidth}{0pt}
}

\setcounter{tocdepth}{2}
\tableofcontents
\clearpage

\chapter{Wstęp}
\label{cha:ch1_wstep}

Celem niniejszej pracy magisterskiej było dobranie algorytmów regulacji oraz samostrojenia dla niestabilnego, nieliniowego obiektu mechatronicznego kontrolowanego przez sterownik PLC. W tym celu, w ramach prac przygotowawczych, od podstaw został wykonany obiekt typu kulka na belce wraz z~zestawem niezbędnych czujników, urządzeniem wykonawczym w formie silnika prądu stałego, układem regulacji opartym o przemysłowy sterownik PLC oraz pozostałymi niezbędnymi elementami i innymi podzespołami pomocniczymi.

Motywacją do podjęcia się tego tematu była chęć zmierzenia się z budową średnio-skomplikowanego układu regulacji od zera, a także chęć zbudowania algorytmów regulacji w oparciu o przemysłowy sterownik PLC. Należy w tym miejscu zauważyć, że większość nieliniowych obiektów regulacji dostępnych w laboratorium sterowania cyfrowego w trakcie cyklu studiów nie była sterowana przy użyciu sterownika PLC, a w przemyśle takie sterowniki stanowią \textit{de facto} standard.

% związek zagadnienia z literaturą
% związek zagadnienia z przemysłem
% Tu motywacje (mogą być też osobiste - zawodowe), uzasadnienie potrzeby, wskazanie na popularność, umocowanie w szerszym konktekście, wykazanie pożytków itp.

\section{Zawartość pracy}

Opis zasady działania układu typu kulka i belka, projekt mechaniczny, budowę obiektu regulacji (konstrukcję belki i przeniesienie napędu) zawarto w rozdziale \ref{cha:ch2_obiekt_regulacji}. W kolejnym rozdziale (\ref{cha:ch3_uklad_ster_i_instrumentacji}) umieszczono opis oprzyrządowania użytego w obiekcie (czujniki odległości, bazowania, sterownik PLC, zespół silnika z enkoderem i przekładnią) oraz okablowania.

Część zasadnicza pracy rozpoczyna się w rozdziale \ref{cha:ch4_model_symulacyjny} od opisu budowy analitycznego modelu symulacyjnego obiektu. Zostało to wykonane przy użyciu narzędzi inżynieryjnych dostępnych w pakiecie oprogramowania \textsc{Matlab/Simulink}, a konkretnie przybornika \textsc{Simscape Multibody}. Sam model obiektu powstał na podstawie fizycznych własności rzeczywistego obiektu: jego wymiarów, ułożenia oraz masy, i posłużył między innymi do aproksymacji zależności kąta obrotu belki od kąta obrotu wału motoreduktora, co było konieczne, gdyż w obiekcie nie użyto czujnika mierzącego kąt obrotu belki. W tym samym rozdziale opisano również budowę, już na podstawie równań matematycznych, modelu silnika wykorzystanego w obiekcie.

Kolejny rozdział (\ref{cha:ch5_identyfikacja}) zawiera opisy przeprowadzonych procesów identyfikacyjnych (kalibracji) czujników odległości oraz parametrów silnika. W przypadku parametrów silnika identyfikacja była kilkustopniowa: najpierw zweryfikowano i poprawiono wartości podane przez producenta, następnie na ich podstawie oraz wykorzystując równania elektryczne i mechaniczne silnika wyliczono kilka parametrów niepodanych przez producenta; w ostatnim kroku zoptymalizowano numerycznie wartości wszystkich parametrów.

W rozdziale \ref{cha:ch6_model_liniowy}, ponownie wykorzystując narzędzia przybornika do projektowania systemów sterowania z pakietu \textsc{Matlab/Simulink}, uzyskano model liniowy całego układu, który następnie poddano przekształceniom tak, aby przyjął formę potrzebną do użycia w zaproponowanym kaskadowym układzie regulacji. Dalej opisano syntezę regulatorów opartych o stan systemu.

Algorytmy sterowania, bazowania i procedury odczytu wartości z czujników zostały opisane w rozdziale \ref{cha:ch7_algorytmy_sterowania}. Znajduje się tam wyszczególnienie głównej sekwencji programu oraz opis implementacji algorytmów. W kolejnym rozdziale opisano procedurę automatycznej identyfikacji (opartej o odpowiedź skokową) systemu odpowiadającego za zachowanie belki; również w tym rozdziale opisana jest próba identyfikacji kulki.

W ostatnim rozdziale zaprezentowano wyniki przeprowadzonych symulacji modeli nieliniowego i~liniowego oraz eksperymentów wykonanych dla rzeczywistego obiektu regulacji. Eksperymenty dotyczyły reakcji belki na zadane położenie przed i po wykonanej procedurze samostrojenia, a także przedstawiały działanie regulatorów w zadaniu stabilizacji położenia kulki przed i po procedurze samo\-strojenia.

W pracy zamieszczono również dwa dodatki opisujące inne możliwe do zastosowania zespoły napędowe (dodatek \ref{appA_warianty_zespolu_napedowego}) i alternatywne czujniki pozycji kulki (dodatek \ref{appB_alternatywne_czujniki_pozycji_kulki}).


%---------------------------------------------------------------------------

\printbibliography
\listoffigures
\listoftables

\end{document}
