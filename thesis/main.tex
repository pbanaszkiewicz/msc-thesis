\documentclass[11pt]{aghdpl}
% \documentclass[en,11pt]{aghdpl}  % praca w języku angielskim

% Lista wszystkich języków stanowiących języki pozycji bibliograficznych użytych w pracy.
% (Zgodnie z zasadami tworzenia bibliografii każda pozycja powinna zostać utworzona zgodnie z zasadami języka, w którym dana publikacja została napisana.)
\usepackage[english,polish]{babel}

% Użyj polskiego łamania wyrazów (zamiast domyślnego angielskiego).
\usepackage{polski}

\usepackage[utf8]{inputenc}

% dodatkowe pakiety

\usepackage{mathtools}
\usepackage{amsfonts}
\usepackage{amsmath}
\usepackage{amsthm}

% operator '\sgn'
\DeclareMathOperator{\sgn}{sgn}

% polecenie \diff generujące \mathrm{d}
\makeatletter
\providecommand*{\diff}%
{\@ifnextchar^{\DIfF}{\DIfF^{}}}
\def\DIfF^#1{%
    \mathop{\mathrm{\mathstrut d}}%
    \nolimits^{#1}\gobblespace}
\def\gobblespace{%
    \futurelet\diffarg\opspace}
\def\opspace{%
    \let\DiffSpace\!%
    \ifx\diffarg(%
    \let\DiffSpace\relax
    \else
    \ifx\diffarg[%
    \let\DiffSpace\relax
    \else
    \ifx\diffarg\{%
    \let\DiffSpace\relax
    \fi\fi\fi\DiffSpace}
    
% polecenie \deriv generujące operator różniczkujący \frac{\diff^{N}x}{\diff{t}^N}
\providecommand*{\deriv}[3][]{%
    \frac{\diff^{#1}#2}{\diff #3^{#1}}}

% polecenie \pderiv generujące operator różniczkujący cząstkowy (przykład jw.)
\providecommand*{\pderiv}[3][]{\frac{\partial^{#1}#2}{\partial #3^{#1}}}

% ------------------------
% --- < bibliografia > ---

\usepackage[
style=numeric,
sorting=none,
%
% Zastosuj styl wpisu bibliograficznego właściwy językowi publikacji.
language=autobib,
autolang=other,
% Zapisuj datę dostępu do strony WWW w formacie RRRR-MM-DD.
urldate=iso8601,
% Nie dodawaj numerów stron, na których występuje cytowanie.
backref=false,
% Podawaj ISBN.
isbn=true,
% Nie podawaj URL-i, o ile nie jest to konieczne.
url=false,
%
% Ustawienia związane z polskimi normami dla bibliografii.
maxbibnames=3,
% Jeżeli używamy BibTeXa:
% backend=bibtex
backend=biber
]{biblatex}

\usepackage{csquotes}
% Ponieważ `csquotes` nie posiada polskiego stylu, można skorzystać z mocno zbliżonego stylu chorwackiego.
\DeclareQuoteAlias{croatian}{polish}

\addbibresource{bibliography.bib}

% Nie wyświetlaj wybranych pól.
%\AtEveryBibitem{\clearfield{note}}


% --------------------
% --- < listingi > ---

% Użyj czcionki kroju Courier.
\usepackage{courier}

\usepackage{listings}
\lstloadlanguages{TeX}

\lstset{
	literate={ą}{{\k{a}}}1
           {ć}{{\'c}}1
           {ę}{{\k{e}}}1
           {ó}{{\'o}}1
           {ń}{{\'n}}1
           {ł}{{\l{}}}1
           {ś}{{\'s}}1
           {ź}{{\'z}}1
           {ż}{{\.z}}1
           {Ą}{{\k{A}}}1
           {Ć}{{\'C}}1
           {Ę}{{\k{E}}}1
           {Ó}{{\'O}}1
           {Ń}{{\'N}}1
           {Ł}{{\L{}}}1
           {Ś}{{\'S}}1
           {Ź}{{\'Z}}1
           {Ż}{{\.Z}}1,
	basicstyle=\footnotesize\ttfamily,
}

% ------------------------

\AtBeginDocument{
	\renewcommand{\tablename}{Tabela}
	\renewcommand{\figurename}{Rys.}
}

% ------------------
% --- < tabele > ---

\usepackage{array}
\usepackage{tabularx}
\usepackage{multirow}
\usepackage{booktabs}
\usepackage{makecell}
\usepackage[flushleft]{threeparttable}
\usepackage{longtable}

% defines the X column to use m (\parbox[c]) instead of p (`parbox[t]`)
\newcolumntype{C}[1]{>{\hsize=#1\hsize\centering\arraybackslash}X}

% ------------------------------
% --- < linki i referencje > ---

\usepackage[hidelinks, colorlinks=false]{hyperref}
\usepackage{cleveref}

\crefname{figure}{rys.}{rys.}

% ------------------
% --- < figury > ---

% najprawdopodobniej jakaś paczka ładuje subcaption, a svg ładuje subfig, które
% nie są kompatybilne; poniższe rozwiązanie wzięte z https://tex.stackexchange.com/a/213279
\usepackage{subcaption}
\expandafter\def\csname ver@subfig.sty\endcsname{}


\usepackage{graphicx}
\usepackage{float} % lepsze pozycjonowanie
\usepackage{svg}
\graphicspath{{graphics/}}  % obrazki i zdjęcia muszą być w podfolderze "graphics"

\usepackage{rotating}

% ----------------------------
% --- < liczby i symbole > ---

\usepackage{siunitx}

\sisetup{
    group-digits = integer,
    list-final-separator = { \translate{and} },
    range-phrase = { -- },
    mode = text,
    range-units = single,
}

% -------------------
% --- < dodatki > ---
%\usepackage[toc,page]{appendix}
%\usepackage[ampersand]{easylist}

%---------------------------------------------------------------------------

\author{inż. Piotr Banaszkiewicz}
\shortauthor{P. Banaszkiewicz}

\titlePL{Dobór algorytmów regulacji oraz samostrojenia dla sterownika PLC współpracującego z nieliniowym obiektem mechatronicznym}
\titleEN{Synthesis of control and self tuning algorithms for a~PLC controlling a~nonlinear mechatronic ball and beam plant}


\shorttitlePL{Dobór algorytmów regulacji oraz samostrojenia dla sterownika PLC współpracującego z~nieliniowym obiektem mechatronicznym}
\shorttitleEN{Synthesis of control and self tuning algorithms for a PLC controlling a~nonlinear mechatronic ball and beam plant}

\thesistype{Praca dyplomowa magisterska}
%\thesistype{Master of Science Thesis}

\supervisor{dr inż. Andrzej Tutaj}
%\supervisor{Marcin Szpyrka PhD, DSc}

\degreeprogramme{Automatyka i Robotyka}
%\degreeprogramme{Computer Science}

\date{2017}

\department{Katedra Automatyki i Inżynierii Biomedycznej}
%\department{Department of Applied Computer Science}

\faculty{Wydział Elektrotechniki, Automatyki,\protect\\[-1mm] Informatyki i Inżynierii Biomedycznej}
%\faculty{Faculty of Electrical Engineering, Automatics, Computer Science and Biomedical Engineering}

\acknowledgements{Serdecznie dziękuję opiekunowi pracy, Panu Doktorowi Andrzejowi Tutajowi, za niesioną pomoc i~zawsze dobrą radę.}


\setlength{\cftsecnumwidth}{10mm}

%---------------------------------------------------------------------------
\setcounter{secnumdepth}{4}
%\brokenpenalty=10000\relax  %?????

\begin{document}

\titlepages

% Ponowne zdefiniowanie stylu `plain`, aby usunąć numer strony z pierwszej strony spisu treści i poszczególnych rozdziałów.
\fancypagestyle{plain}
{
	% Usuń nagłówek i stopkę
	\fancyhf{}
	% Usuń linie.
	\renewcommand{\headrulewidth}{0pt}
	\renewcommand{\footrulewidth}{0pt}
}

\setcounter{tocdepth}{2}
\tableofcontents
\clearpage

\chapter{Wstęp}
\label{cha:ch1_wstep}

Celem niniejszej pracy magisterskiej było dobranie algorytmów regulacji oraz samostrojenia dla niestabilnego, nieliniowego obiektu mechatronicznego kontrolowanego przez sterownik PLC. W tym celu, w ramach prac przygotowawczych, od podstaw został wykonany obiekt typu kulka na belce wraz z~zestawem niezbędnych czujników, urządzeniem wykonawczym w formie silnika prądu stałego, układem regulacji opartym o przemysłowy sterownik PLC oraz pozostałymi niezbędnymi elementami i innymi podzespołami pomocniczymi.

Motywacją do podjęcia się tego tematu była chęć zmierzenia się z budową średnio-skomplikowanego układu regulacji od zera, a także chęć zbudowania algorytmów regulacji w oparciu o przemysłowy sterownik PLC. Należy w tym miejscu zauważyć, że większość nieliniowych obiektów regulacji dostępnych w laboratorium sterowania cyfrowego w trakcie cyklu studiów nie była sterowana przy użyciu sterownika PLC, a w przemyśle takie sterowniki stanowią \textit{de facto} standard.

% związek zagadnienia z literaturą
% związek zagadnienia z przemysłem
% Tu motywacje (mogą być też osobiste - zawodowe), uzasadnienie potrzeby, wskazanie na popularność, umocowanie w szerszym konktekście, wykazanie pożytków itp.

\section{Zawartość pracy}

Opis zasady działania układu typu kulka i belka, projekt mechaniczny, budowę obiektu regulacji (konstrukcję belki i przeniesienie napędu) zawarto w rozdziale \ref{cha:ch2_obiekt_regulacji}. W kolejnym rozdziale (\ref{cha:ch3_uklad_ster_i_instrumentacji}) umieszczono opis oprzyrządowania użytego w obiekcie (czujniki odległości, bazowania, sterownik PLC, zespół silnika z enkoderem i przekładnią) oraz okablowania.

Część zasadnicza pracy rozpoczyna się w rozdziale \ref{cha:ch4_model_symulacyjny} od opisu budowy analitycznego modelu symulacyjnego obiektu. Zostało to wykonane przy użyciu narzędzi inżynieryjnych dostępnych w pakiecie oprogramowania \textsc{Matlab/Simulink}, a konkretnie przybornika \textsc{Simscape Multibody}. Sam model obiektu powstał na podstawie fizycznych własności rzeczywistego obiektu: jego wymiarów, ułożenia oraz masy, i posłużył między innymi do aproksymacji zależności kąta obrotu belki od kąta obrotu wału motoreduktora, co było konieczne, gdyż w obiekcie nie użyto czujnika mierzącego kąt obrotu belki. W tym samym rozdziale opisano również budowę, już na podstawie równań matematycznych, modelu silnika wykorzystanego w obiekcie.

Kolejny rozdział (\ref{cha:ch5_identyfikacja}) zawiera opisy przeprowadzonych procesów identyfikacyjnych czujników odległości oraz parametrów silnika. W przypadku parametrów silnika identyfikacja była kilkustopniowa: najpierw zweryfikowano i poprawiono wartości podane przez producenta, następnie na ich podstawie oraz wykorzystując równania elektryczne i mechaniczne silnika wyliczono kilka parametrów niepodanych przez producenta; w ostatnim kroku zoptymalizowano numerycznie wartości wszystkich parametrów.

W rozdziale \ref{cha:ch6_model_liniowy}, ponownie wykorzystując narzędzia przybornika do projektowania systemów sterowania z pakietu \textsc{Matlab/Simulink}, uzyskano model liniowy całego układu, który następnie poddano przekształceniom tak, aby przyjął formę potrzebną do użycia w zaproponowanym kaskadowym układzie regulacji. Dalej opisano syntezę regulatorów opartych o stan systemu.

Algorytmy sterowania, bazowania i procedury odczytu wartości z czujników zostały opisane w rozdziale \ref{cha:ch7_algorytmy_sterowania}. Znajduje się tam wyszczególnienie głównej sekwencji programu oraz opis implementacji algorytmów. W kolejnym rozdziale opisano procedurę identyfikacji (opartą o odpowiedź skokową) systemu odpowiadającego za zachowanie belki; również w tym rozdziale opisana jest próba identyfikacji kulki.

W ostatnim rozdziale zaprezentowano wyniki przeprowadzonych symulacji modeli nieliniowego i~liniowego oraz eksperymentów zaczerpniętych z rzeczywistego obiektu regulacji. Eksperymenty dotyczyły reakcji belki na zadane położenie przed i po wykonanej procedurze samostrojenia, a także przedstawiały działanie regulatorów w zadaniu stabilizacji położenia kulki przed i po procedurze samostrojenia.

W pracy zamieszczono również dwa dodatki opisujące inne możliwe do zastosowania zespoły napędowe (dodatek \ref{appA_warianty_zespolu_napedowego}) i alternatywne czujniki pozycji kulki (dodatek \ref{appB_alternatywne_czujniki_pozycji_kulki}).


%---------------------------------------------------------------------------
\chapter{Obiekt regulacji}
\label{cha:ch2_obiekt_regulacji}

Obiektem poddawanym regulacji był system typu kulka na belce, który został zbudowany od podstaw na potrzeby tej pracy.

%%%%%%%%
\section{Obiekty typu kulka na belce}

Na system tego typu składają się długa, umieszczona horyzontalnie belka i silnik lub serwomechanizm, który umożliwia wychylanie się belki.
Po belce swobodnie porusza się kulka.

Podstawowym zadaniem regulacji w systemie tego typu jest stabilizacja położenia kulki w wybranym punkcie.
Charakterystyczną cechą tego systemu jest prostota konstrukcji oraz niestabilność przy braku aktywnej regulacji.

Obiekty tego typu są często wykorzystywane w dydaktyce teorii sterowania. Składają się na to poniższe powody:

\begin{itemize}
	\item prostota budowy,
	\item możliwość zastosowania różnych czujników położenia kulki,
	\item możliwość zastosowania różnych silników i mechanizmów przeniesienia napędu.
\end{itemize}

Uproszczony schemat systemu kulka i belka przedstawiony został poniżej:

\begin{figure}[H]
	\centering
	\includesvg[width=0.7\textwidth,svgpath=./graphics/]{kulka_belka_schemat_uproszczony}
	\caption{Uproszczony schemat systemu typu kulka i belka.}
	\label{fig:kulka_belka_schemat_uproszczony}
\end{figure}

% TODO: dodać źródła
Prostota konstrukcji i inherentna niestabilność sprawiły, że powstało wiele implementacji tego systemu [1][2][3][4], również komercyjne, jak na przykład produkt firmy Quanser:

\begin{figure}[H]
	\centering
	\includegraphics[width=0.5\textwidth]{quanser_ball_beam}
	\caption{Zdjęcie produktu \textit{Ball and Beam} firmy Quanser.}
	\label{fig:quanser_ball_beam}
% TODO: dodać źródło
\end{figure}

%%%%%%%%
\section{Projekt mechaniczny}

Przed przystąpieniem do budowy obiektu, zaprojektowano wstępny kształt w programie \textsc{CAD} \texttt{SketchUp Make} (\cref{fig:cad_render_1}). Wyszczególniono na nim:

\begin{itemize}
	\item prostokątną podstawę,
	\item słupy podtrzymujące belkę,
	\item usztywniający łącznik między słupami,
	\item oś obrotu (wał) umieszczony w połowie długości belki,
	\item przekrój belki.
\end{itemize}

\begin{figure}[H]
	\centering
	\includegraphics[width=0.4\textwidth]{cad}
	\caption{Render projektu CAD.}
	\label{fig:cad_render_1}
\end{figure}

Ostateczna konstrukcja różni się względem projektu \textsc{CAD} o wysokość słupów, i umiejscowienie łącznika między nimi. Dodatkowo zastosowano sztywne połączenie osi obrotu belki i samej belki wykorzystujące podpory wału.

%%%%%%%%
\section{Konstrukcja mechaniczna}

Większość konstrukcji powstała z ocynkowanych elementów stalowych, tzw. ,,ceowników'' w przekroju kwadratowym o boku długości \SI[mode=text]{4}{cm}, pozwalających na łatwe skręcanie kilku elementów ze sobą. Rozwiązanie to jest bardzo tanie w porównaniu do przemysłowych profili aluminiowych lub spawanych profili stalowych, ale jednocześnie jest dość ciężkie i poprzez niedomknięcie profilu podatne na pewne momenty gnące.

Kąty proste pomiędzy elementami ustawionymi prostopadle zostały zapewnione poprzez skręcenie za pomocą wsporników stalowych.

\begin{figure}[H]
	\centering
	\includegraphics[width=0.7\textwidth]{mechatronics_perspective_1}
	\caption{Zdjęcie obiektu regulacji w trakcie budowy.}
	\label{fig:mechatronics_perspective_1}
\end{figure}

Na prostokątnej podstawie o wymiarach zewnętrznych \SI[mode=text]{60 x 23}{cm} wykonanej z ceowników ustawiono pionowo na środkach dłuższych boków słupy nośne, również wykonane z ceowników. Na słupach przyczepiono współosiowo łożyska maszynowe samonastawne typu UCFL 201 w obudowach odlewanych.

Słupy zostały usztywnione poprzez połączenie ich przęsłem podniesionym o \SI[mode=text]{11}{cm} względem podstawy.

Przez łożyska poprowadzono pręt nierdzewny stalowy o średnicy \SI[mode=text]{12}{mm}; na pręt nałożono podpory wałka w kształcie litery \texttt{T}, a do nich przykręcono belkę.

Silnik elektryczny, przymocowany do aluminiowego uchwytu, został umieszczony podłużnie na krótszym boku podstawy, na podwyższeniu wykonanym z dwóch elementów stalowych typu ceownik.

% TODO: podwyższenie zostało wzmocnione przeciwko gięciom poprzecznym i wzdłużnym poprzez zastosowanie …………

%%%%%%%%
\section{Przeniesienie napędu}

W obiekcie zastosowano przeniesienie napędu wykorzystujące mechanizm korbowy. Rozwiązanie to posiada kilka zalet:

\begin{itemize}
	\item gwarantuje bezpieczeństwo mechanizmu -- błąd algorytmiczny (np. przypadkowe podanie maksymalnego sterowania) nie spowoduje uszkodzenia fizycznego żadnej części obiektu,
	\item poprzez oddalenie punktu zaczepu korbowodu od osi obrotu belki zmniejsza wymagania dotyczące mocy silnika, a tym samym jego cenę,
	\item pozwala regulować zakres wychyleń belki w wyniku zmiany długości korby.
\end{itemize}

% TODO: napraw kolory w schemacie
% TODO: dodaj opis (korba, korbowód)
\begin{figure}[H]
	\centering
	\includesvg[width=0.6\textwidth,svgpath=./graphics/]{schemat_korba}
	\caption{Schemat napędu opartego o mechanizm korbowy.}
	\label{fig:schemat_korba}
\end{figure}

%%%%%%%%
\section{Belka}

Belka została stworzona poprzez trwałe sklejenie krawędzi kątownika aluminiowego o długości \SI[mode=text]{40}{cm} i boku \SI[mode=text]{3}{cm} oraz krawędzi ceownika aluminiowego o długości \SI[mode=text]{65}{cm} i boku \SI[mode=text]{4}{cm}. W przekroju przypomina to kształtem literę \texttt{M} domkniętą od spodu (\cref{fig:przekroj_belki}).

\begin{figure}[H]
	\centering
	\includesvg[width=0.2\textwidth,svgpath=./graphics/]{beam_xsection}
	\caption{Schemat przekroju belki z zaznaczonymi ceownikiem aluminiowym i kątownikiem aluminiowym.}
	\label{fig:przekroj_belki}
\end{figure}

Krótszy kątownik został umieszczony wzdłużnie na środku belki. W odległości około \SI[mode=text]{1}{cm} od jego końców zamontowano uchwyty na czujniki optyczne.

Uchwyty pozwalają na zmianę wysokości czujnika względem płaszczyzny belki, a także na pochylenie go w osi prostopadłej do płaszczyzny belki.

% TODO: schemat / render bracketu na czujnik

%%%%%%%%
\section{Kulka}

Do projektu dobrano lekką kulkę o masie \SI[mode=text]{20}{g} wykonaną z miękkiej gąbki; średnica kulki wynosi \SI[mode=text]{6}{cm}.

%---------------------------------------------------------------------------
\chapter{Układ sterowania i instrumentacji}
\label{cha:ch3_uklad_ster_i_instrumentacji}

Do odczytywania danych z obiektu i sterowania nim wykorzystano opisany w tym rozdziale układ sterowania i instrumentacji (\cref{fig:schemat_ukl_sterowania_instrumentacji}). W jego sercu znajduje się przemysłowy sterownik PLC, który odczytuje dane o położeniu kulki z dwóch czujników odległości oraz położenie kątowe wału silnika z~enkodera. Dodatkowo do sterownika podłączony został czujnik bazowania oraz przyciski: \texttt{START} (NO), \texttt{STOP} (NC). Na wyjścia sterownika podłączony został mostek H kontrolujący silnik oraz dioda sygnalizacyjna.

% TODO: zweryfikować podłączenie przycisków

\begin{figure}[H]
    \centering
    % TODO: przerobić rysunek w Inkscape
    \includegraphics[width=0.7\textwidth]{schemat_ukladu3}
    \caption{Schemat układu sterowania i instrumentacji wraz z zaznaczonymi połączeniami.}
    \label{fig:schemat_ukl_sterowania_instrumentacji}
\end{figure}

%%%%%%%%
\section{Sterownik PLC}
\label{sec:ch3_PLC}

Wykorzystany w pracy sterownik PLC to Siemens S7-1211C DC/DC/DC. Działa on na napięciu stałym \SI{24}{V}, posiada 6 wejść dyskretnych \SI{24}{V}, 4 tranzystorowe wyjścia dyskretne \SI{24}{V} i 2 napięciowe wejścia analogowe \SIrange[range-units=single]{0}{10}{V}.

W celu ułatwienia komunikacji między sterownikiem i elektroniką opartą o logikę \SI{5}{V} (więcej w podrozdziale \ref{sec:ch3_systemy_napiec}), został on rozszerzony o dodatkową płytkę sygnałową SB 1223 działającą na logice \SI{5}{V}; dodaje ona po 2 wejścia i~wyjścia dyskretne \SI{5}{V}.

Podstawowe parametry sterownika oraz płytki sygnałowej zostały zebrane w tabeli \ref{tab:parametry_PLC_SB}:

\begin{table}[H]
    \centering
    \begin{threeparttable}
        \caption{Podstawowe parametry sterownika PLC Siemens S7-1211C i płytki sygnałowej Siemens SB 1223\tnote{a}.}
        \label{tab:parametry_PLC_SB}
        
        \begin{tabularx}{\textwidth}{p{5cm} | p{5cm} | p{5cm} }
            \toprule
            Nazwa & Siemens S7-1211C & Siemens SB 1223 \\
            \midrule
            Napięcie zasilania & \SI{24}{V} DC & \SI{5}{V} DC \\
            \midrule
            Ilość wejść cyfrowych & 6 & 2 \\
            Ilość wyjść cyfrowych & 4 & 2 \\
            Ilość wejść analogowych & 2 & 0 \\
            Ilość wyjść analogowych & 0 & 0 \\
            Typ wejść cyfrowych & \textit{sink-source} & \textit{source} \\
            Typ wyjść cyfrowych & półprzewodnikowe MOSFET \textit{source} & półprzewodnikowe MOSFET \textit{sink-source} \\
            Typ wejść analogowych & Napięciowe \SIrange[range-units=single]{0}{10}{V} & n.d. \\
            \midrule
            Szybkie liczniki & Do 6 z częstotliwością \SI{100}{kHz}\tnote{b} & Do 2 z częstotliwością \SI{200}{kHz}\tnote{c} \\
            Wyjścia impulsowe & Do 4 z częstotliwością \SI{100}{kHz} & Do 2 z częstotliwością \SI{200}{kHz} \\
            \midrule
            Pamięć robocza & \SI{30}{kB} & n.d. \\
            Pamięć ładowania & \SI{1}{MB} & n.d. \\
            Pamięć trwała & \SI{10}{kB} & n.d. \\
            \midrule
            Czas wykonywania instrukcji boolowskich & \SI{0,08}{\micro\second}/instrukcję & n.d. \\
            Czas wykonywania operacji na typie WORD & \SI{1,7}{\micro\second}/instrukcję & n.d. \\
            Czas wykonywania operacji na typie REAL & \SI{2,3}{\micro\second}/instrukcję & n.d. \\
            \bottomrule
        \end{tabularx}
        
        \begin{tablenotes}
            \footnotesize
            \item[a] opracowanie własne na podstawie \cite{S7MANUAL},
            \item[b] w trybie kwadraturowym wykorzystywane są dwa wejścia, a~maksymalna częstotliwość wynosi \SI{80}{kHz},
            \item[c] w trybie kwadraturowym wykorzystywane są dwa wejścia, a~maksymalna częstotliwość wynosi \SI{160}{kHz}.
        \end{tablenotes}
    \end{threeparttable}
\end{table}

% TODO: dodaj zdjęcia

%%%%%%%%
\section{Silnik z reduktorem i enkoderem}
\label{sec:ch3_uklad_napedowy}

Jak już zasygnalizowano w rozdziale \ref{sec:ch2_przeniesienie_napedu}, w pracy użyto silnika prądu stałego (komutatorowy, z~magnesami trwałymi). Silnik sprzężony jest z~zębatą przekładnią redukcyjną o przełożeniu \num{18,75}:\num{1}. Za silnikiem umieszczony jest enkoder inkrementalny kwadraturowy o~64 impulsach na obrót wału, co daje 1200 impulsów za przekładnią.

Wybrany silnik stanowi dobry kompromis między złożonością, wydajnością i ceną. Dyskusja na temat możliwości zastosowania innych typów napędów została przeprowadzona w dodatku \ref{appA_warianty_zespolu_napedowego}.

Producent silnika nie dostarcza pełnej dokumentacji, a jedynie kilka wybranych parametrów. Wymusiło to analityczne lub eksperymentalne wyznaczenie pozostałych wymaganych do zamodelowania silnika parametrów. Wszystkie parametry zostały przedstawione w tabeli \ref{tab:parametry_silnika} poniżej.

\begin{table}[h]
    \centering
    \begin{threeparttable}
        \caption{Parametry fizyczne, elektryczne i mechaniczne silnika, enkodera i przekładni\tnote{a}.}
        \label{tab:parametry_silnika}
        
        \begin{tabularx}{0.6\textwidth}{l | l}
            \toprule
            Średnica & \SI{37}{\milli\meter} \\
            Długość & \SI{68}{\milli\meter} \\
            Masa & \SI{215}{g} \\
            Średnica wału & \SI{6}{\milli\meter} \\
            \midrule
            Przełożenie przekładni & \num{18,75}:\num{1} \\
            \midrule
            Napięcie znamionowe & \SI{12}{\volt} \\
            Prędkość znamionowa & \SI{52,36}{\radian\per\second} \\
            Prąd znamionowy & \SI{300}{\milli\ampere} \\
            Prąd zatrzymania silnika & \SI{5000}{\milli\ampere} \\
            Moment zatrzymania silnika & \SI{0,59}{\newton\meter} \\
            \midrule
            Rezystancja\tnote{b} & \SI{2,4}{\ohm} \\
            Stała SEM rotacji\tnote{b} $K_e$ & \SI{0,2154}{\volt\per\radian\per\second} \\
            Stała momentu\tnote{b} $K_t$ & \SI{0,2154}{\newton\meter\per\ampere} \\
            Współczynnik tarcia wiskotycznego\tnote{b} $\beta$ & \num{0,00161} \\
            Współczynnik tarcia suchego\tnote{b} $b$ & \num{0,019} \\
            Moment bezwładności przekładni i wału\tnote{c} $J$ & \num{0,00123} \\
            \bottomrule
        \end{tabularx}
        
        \begin{tablenotes}
            \footnotesize
            \item[a] opracowanie własne na podstawie \cite{SILNIK_MANUAL},
            % TODO: zaktualizuj odnośniki
            \item[b] zidentyfikowano analitycznie, więcej w rozdziale \ref{cha:ch6_identyfikacja},
            \item[b] zidentyfikowano eksperymentalnie, więcej w rozdziale \ref{cha:ch6_identyfikacja}.
        \end{tablenotes}
    \end{threeparttable}
\end{table}

Silnik sterowany jest przez PLC za pomocą układu mostka H. Został on dobrany tak, by spełniać wymagania elektryczne silnika w pracy znamionowej. Sterowany jest sygnałem PWM o częstotliwości przenoszenia \SI{20}{\kilo\hertz}. Dodatkowym sygnałem jest binarny sygnał kierunku obrotu silnika.

% TODO: sposób wyznaczania kąta pochylenia belki, mostek H - parametry

%%%%%%%%
\section{Czujniki odległości}
\label{sec:ch3_czujniki_odleglosci}

% TODO: opis czujników, charakterystyki, położenie, sposób wyznaczania pozycji

%%%%%%%%
\section{Czujnik bazowania}
\label{sec:ch3_czujnik_bazowania}

% TODO: transoptor szczelinowy, sposób montażu, schemat połączeń

%%%%%%%%
\section{Okablowanie i zabezpieczenia}
\label{sec:ch3_okablowanie_zabezpieczenia}

% TODO: zdjęcie "szafy"

%%%%%%%%
\section{Systemy napięć}
\label{sec:ch3_systemy_napiec}

% TODO: 5V, 12V, 24V

%%%%%%%%
\section{Podsumowanie}


%---------------------------------------------------------------------------
\chapter{Model symulacyjny}
\label{cha:ch4_model_symulacyjny}

Do zaprojektowania odpowiedniego układu regulacji systemem kulka i belka, a więc dobrania struktury regulatorów, konieczne jest posiadanie modelu tego systemu. Jest to jednak zadanie utrudnione, gdy system jest wybitnie nieliniowy --- a obiekt regulacji taki właśnie jest. Mamy tutaj do czynienia z~nieliniowościami wynikającymi z:
\begin{itemize}
    \item zastosowania silnika prądu stałego,
    \item przeniesienia napędu poprzez przekładnię korbową,
    \item czy z dynamiki kulki zależnej od ruchu obrotowego belki.
\end{itemize}

% TODO: uaktualinij odnośnik
Wobec tego zdecydowano się nie przystępować do opisu matematycznego zachowania się układu w sposób klasyczny, np. poprzez równania Eulera--Lagrange'a czy funkcjonał Hamiltona. Zamiast tego, wykorzystując narzędzia pakietu \textsc{Matlab/Simulink}, zbudowano przestrzenną i fizyczną reprezentację obiektu regulacji, która następnie została wykorzystana do linearyzacji i dobrania regulatorów (rozdział \ref{cha:ch7_algorytmy_sterowania}).

\section{Wykorzystanie przybornika \textsc{SimMechanics}}
\label{sec:ch4_simmechanics}

\textsc{SimMechanics}, znany również jako \textsc{Simscape Multibody}, pozwala na modelowanie fizycznych układów (ciał stałych o określonej geometrii, masie i/lub inercji) oraz zależności między nimi (transformacje i rotacje pozycji). Bardzo ważną cechą jest możliwość stosowania więzów pomiędzy kilkoma układami odniesienia. Więzy te mają zero lub więcej stopni swobody i nazywane są połączeniami lub złączami; do najbardziej charakterystycznych należą złącza pryzmatyczne, obrotowe czy sferyczne (zob. \cref{fig:simmechanics_bloki}). Jak można zauważyć, odpowiadają one fizycznym połączeniom ruchowym lub obrotowym spotykanym w układach mechanicznych.

\begin{figure}[h]
    \centering
    \begin{subfigure}[t]{0.15\textwidth}
        \centering
        \includegraphics[width=0.75\textwidth]{sm_solid}
        \caption{Ciało stałe.}
        \label{fig:sm_solid}
    \end{subfigure}
    ~ 
    \begin{subfigure}[t]{0.15\textwidth}
        \centering
        \includegraphics[width=0.75\textwidth]{sm_rigid_transform}
        \caption{Transformacja układu odniesienia.}
        \label{fig:sm_rigid_transform}
    \end{subfigure}
    ~ 
    \begin{subfigure}[t]{0.15\textwidth}
        \centering
        \includegraphics[width=0.75\textwidth]{sm_prismatic_joint}
        \caption{Złącze pryzmatyczne.}
        \label{fig:sm_prismatic_joint}
    \end{subfigure}
    ~
    \begin{subfigure}[t]{0.15\textwidth}
        \centering
        \includegraphics[width=0.75\textwidth]{sm_revolute_joint}
        \caption{Złącze obrotowe.}
        \label{fig:sm_revolute_joint}
    \end{subfigure}
    
    \caption{Podstawowe bloki budujące schemat \textsc{SimMechanics}.}
    \label{fig:simmechanics_bloki}
\end{figure}

Budowanie schematu z wykorzystaniem bloków \textsc{SimMechanics} polega na, w dużym uproszczeniu, przyłączaniu ciał stałych w centrach układów odniesienia, które przemieszczane są w przestrzeni za pomocą bloków \textit{Rigid Transform} (fig. \ref{fig:sm_rigid_transform}). Wspomniane bloki umożliwiają również rotację układów odniesienia, co jest często konieczne do poprawnego działania złącz o pewnej ilości stopni swobody. Przykładowo, złącze \textit{Revolute Joint} umożliwia obracanie układu odniesienia \texttt{F} (ang. \textit{Follower} --- następujący układ odniesienia) wokół osi Z~układu \texttt{B} (ang. \textit{Base} --- poprzedzający układ odniesienia)\footnote{W ustawieniach bloku \textit{Revolute Joint} możliwa jest zmiana na tryb odwrotny, tj. obrót układu \texttt{B} wokół układu \texttt{F}.}. Jednakże jeśli wał, który ma się obracać wokół swojej naturalnej osi obrotu, nie ma w swoim układzie osi Z~wzdłuż naturalnej osi obrotu, wtedy jego układ odniesienia musi zostać transformowany za pomocą \textit{Rigid Transform}.

Bloki złącz wspierają informowanie o aktualnych wartościach pozycji, prędkości i przyspieszenia (w zależności od typu złącza odpowiednio liniowych lub obrotowych) oraz przyłożonego na złącze momentu. Dodatkowo możliwe jest załączenie wejścia momentu oddziałującego, co zostało wykorzystane przy implementacji silnika.

Ostatnią wartą wspomnienia informacją jest możliwość ustawienia pewnej wartości wstępnej dla danego złącza. W przypadku złącza \textit{Revolute Joint} umożliwia to wymuszenie kąta obrotu pomiędzy układami odniesienia \texttt{B} oraz \texttt{F}, a w przypadku złącza \textit{Prismatic Joint} wymusza to odpowiednie przesunięcie.

\section{Reprezentacja obiektu typu kulka i belka}
\label{sec:ch4_reprezentacja_obiektu_kulka_i_belka}

Bardzo dużą zaletą korzystania z \textsc{SimMechanics} jest automatyczne generowanie podglądu (zob. \cref{fig:uklad_simmechanics}) budowanego układu. Pozwala to przeprowadzać obserwację działania różnych algorytmów sterowania oraz łatwo znajdować błędy złożenia brył.

\begin{figure}[H]
    \centering
    \includegraphics[width=0.5\textwidth]{simmechanics_ballandbeam}
    \caption{Wizualizacja układu kulki i belki zamodelowanego przy pomocy \textsc{SimMechanics} (por. z obiektem rzeczywistym przedstawionym na \cref{fig:perspektywa}).}
    \label{fig:uklad_simmechanics}
\end{figure}

Schemat użyty do wygenerowania modelu z \cref{fig:uklad_simmechanics} został przedstawiony na \cref{fig:uklad_simmechanics2}. Poczynając od lewej strony, możemy na nim wyróżnić kilka charakterystycznych elementów:

\begin{itemize}
    \item obiekty \textit{Base}, \textit{Tower1}, \textit{Tower2}, \textit{MotorBase}, \textit{Shaft Supports}, \textit{Motor}, \textit{Beam}, \textit{Pin2}, \textit{Ball} oraz \textit{CrankShaft},
    \item subsystemy \textit{Shaft}, \textit{Electric Motor}, \textit{Crank} oraz \textit{Ball mechanics},
    \item sporo bloków \textit{Rigid Transform} o nazwach \textit{RT}---\textit{RT19},
    \item trzy bloki \textit{Revolute Joint},
    \item wejścia \textit{voltage} oraz \textit{disturbance},
    \item wyjścia \textit{ball\_position}, \textit{ball\_velocity}, \textit{beam\_angle} oraz \textit{beam\_angular\_velocity}.
\end{itemize}

Obiekt \textit{Base} reprezentuje podstawę, na której umieszczono całą mechanikę układu. Poprzez transformacje \textit{RT} oraz \textit{RT1} umieszczono na nim dwie wieże, na których zaczepione są łożyska i wał obrotowy belki (subsystem \textit{Shaft}, \cref{fig:sm_shaft}).

Wał belki składa się z dwóch równoległych ścieżek, transformowanych z układu odniesienia świata poprzez bloki \textit{RT2} oraz \textit{RT3}. Za blokami umieszczono złącza obrotowe \textit{BallBearing1} oraz \textit{BallBearing2}, reprezentujące fizyczne łożyska kulkowe (zob. rozdział \ref{sec:ch2_konstrukcja_mechaniczna}). Złącze \textit{BallBearing2} zostało wykorzystane do pobrania z układu aktualnego kąta oraz prędkości kątowej obrotu belki.

Idąc dalej, w odpowiednim przesunięciu od środka wału belki (\textit{RT6}) zamontowano podpory wału (\textit{Shaft Supports}) --- tutaj zamodelowane jako jeden obiekt. Na nich (\textit{RT7}) została przymocowana belka, której kształt w przekroju przypomina domkniętą i wypełnioną literę \texttt{M}.

Na schemacie układu \ref{fig:uklad_simmechanics2} równolegle do ,,górnej'' części obiektu (wał belki, belka, kulka) poprowadzona jest ścieżka silnika i przekładni korbowej; obie ścieżki złączone są poprzez korbowód (\textit{Crankshaft}).

\begin{sidewaysfigure}[p!]
    \centering
    \includegraphics[width=\textwidth]{simmechanics_ballandbeam2}
    \caption{Schemat \textsc{SimMechanics} układu kulki i belki.}
    \label{fig:uklad_simmechanics2}
\end{sidewaysfigure}

Obiekty \textit{MotorBase} oraz \textit{Motor} reprezentują bryły podstawy, na której osadzony jest silnik, oraz silnika. Elementy te nie biorą udziału w ogólnym zachowaniu belki pod wpływem momentu generowanego przez silnik, wobec czego nie nadano im skomplikowanych kształtów, jakie te elementy mają w~rzeczywistości.

\begin{figure}[h]
    \centering
    \includegraphics[width=0.8\textwidth]{simmechanics_shaft}
    \caption{Schemat subsystemu \textit{Shaft} odpowiadającego za umocowanie wału obrotu belki w danym układzie odniesienia.}
    \label{fig:sm_shaft}
\end{figure}

Sposób wykorzystania złącza obrotowego \textit{RJ3} został opisany w rozdziale \ref{sec:ch4_model_silnika}. Za wspomnianym złączem znajduje się subsystem \textit{Crank} (przedstawiony na \cref{fig:sm_shaft}), który składa się głównie z dwóch czarnych pinów (elementów mocujących o zerowej masie --- masa fizycznych przegubów włączona jest do masy korby bądź korbowodu) i korby (\textit{Crank}).

\begin{figure}[h]
    \centering
    \includegraphics[width=0.8\textwidth]{simmechanics_crank}
    \caption{Schemat subsystemu \textit{Crank} odpowiadającego za umocowanie wału obrotu belki w danym układzie odniesienia.}
    \label{fig:sm_crank}
\end{figure}

Ostatnim, nieopisanym dotąd subsystemem, jest \textit{Ball mechanics} --- subsystem połączony z belką (\cref{fig:sm_ball_mechanics}), którego zadaniem jest ,,przytwierdzenie'' kulki do belki; osiągnięto to za pomocą nieopisanego dotąd bloku \textit{Rack and pinion} (pol. \textit{przekładnia zębata współpracująca z kołem zębatym}).

Blok \textit{Rack and pinion} jest specjalnym rodzajem złącza będącym połączeniem złącz pryzmatycznego i obrotowego. W efekcie pracuje on jako więzadło ruchu obrotowo-postępowego, z zerowym tarciem i~z~brakiem oderwania pod wpływem przyspieszeń pionowych, co dobrze przybliża zachowanie kulki w~układzie kulki i belki.

Połączenie równoległe bloków \textit{RPC1} i \textit{PJ}, \textit{RJ} przedstawione na \cref{fig:sm_ball_mechanics} spowodowane jest wymaganiami bloku \textit{Rack and pinion} co do osi przesuwu i osi obrotu; z tego powodu konieczne było użycie odpowiednich transformacji układów odniesienia \textit{RT12} oraz \textit{RT11}.

\begin{figure}[h]
    \centering
    \includegraphics[width=0.9\textwidth]{simmechanics_ball_mechanics}
    \caption{Schemat subsystemu \textit{Ball mechanics} odpowiadającego za zachowanie kulki na belce.}
    \label{fig:sm_ball_mechanics}
\end{figure}

W układzie \cref{fig:uklad_simmechanics2} jedno z wejść, oznaczone jako \textit{disturbance}, przeznaczone jest na symulowanie zewnętrznych zakłóceń wpływających bezpośrednio na kulkę; widoczne jest to w schemacie \textit{Ball mechanics} jako wejście \textit{ball\_pos\_dist}.

Z subsystemu \textit{Ball mechanics} wychodzą również dwa wyjścia: \textit{ball\_position} oraz \textit{ball\_velocity}. Odczytywane są one ze złącza pryzmatycznego kulki \textit{PJ}.

Sama kulka została dołączona do schematu poza subsystemem \textit{Ball mechanics}, jako ciało stałe \textit{Ball}. Równolegle dołączony blok \textit{BallFrame} odpowiada za wyświetlenie wersorów układu odniesienia środka kulki, co pomaga zobaczyć na wizualizacji faktyczny ruch obrotowy kulki po belce.

\section{Model silnika}
\label{sec:ch4_model_silnika}

Widoczny na \cref{fig:sm_electric_motor} model silnika prądu stałego odpowiada modelowi matematycznemu opartemu o równanie elektryczne obwodu silnika \eqref{eq:silnik_r_el} i mechaniczne \eqref{eq:silnik_r_mech}:

\begin{equation}\label{eq:silnik_r_el}
    u - R i - L \deriv{i}{t} = K \omega 
\end{equation}
\begin{equation}\label{eq:silnik_r_mech}
    T = K i - J \epsilon - \beta \omega - b \sgn \omega
\end{equation}
gdzie:
\begin{itemize}
    \item $u$ --- napięcie sterujące,
    \item $R$ --- rezystancja silnika,
    \item $i$ --- prąd w obwodzie twornika,
    \item $K$ --- stała elektryczna bądź mechaniczna silnika (w jednostkach \si{\volt\second\per\radian} lub \si{\newton\meter\per\ampere}),
    \item $L$ --- indukcyjność silnika (w modelu na \cref{fig:sm_electric_motor} przyjęta za zerową),
    \item $T$ --- moment generowany przez silnik,
    \item $J$ --- moment bezwładności wału silnika,
    \item $\omega$ --- prędkość kątowa wału silnika,
    \item $\epsilon = \deriv{\omega}{t}$ --- przyspieszenie kątowe wału silnika,
    \item $\beta$ --- współczynnik tarcia wiskotycznego przekładni oraz silnika,
    \item $b$ --- współczynnik tarcia suchego przekładni oraz silnika.
\end{itemize}

\begin{figure}[h]
    \centering
    \includegraphics[width=0.8\textwidth]{simmechanics_electric_motor.png}
    \caption{Schemat subsystemu \textit{Electric Motor} odpowiadającego za model silnika prądu stałego.}
    \label{fig:sm_electric_motor}
\end{figure}

Warto zauważyć, że obliczanie kąta obrotu, prędkości kątowej oraz przyspieszenia kątowego odbywa się poza blokiem \textit{Electric Motor} --- wykonuje to \textsc{SimMechanics}, biorąc pod uwagę dynamikę reszty układu oraz moment, którym silnik działa na cały mechanizm.

W modelu silnika pominięto indukcyjność, gdyż jej wartość w rzeczywistości jest bardzo niska. Dodatkowo, przeprowadzona identyfikacja tarcia (rozdział \ref{sec:ch6_identyfikacja_parametrow_silnika}) pokazała, że poza tarciem mokrym w~układzie występuje pewne tarcie suche. Z powodu pracy silnika w sposób powtarzalny, tarcie suche może mieć spory wpływ na odpowiedź silnika, dlatego zdecydowano się go nie pomijać.

\section{Aproksymacja zależności kąta obrotu wału silnika i osi belki}
\label{sec:ch4_zaleznosc_kata_silnika_i_kata_belki}

W związku z brakiem czujnika obrotu wału belki, do odczytania jej położenia należy się posłużyć zależnością geometryczną od obrotu wału silnika. W rozdziale \ref{sec:ch3_uklad_napedowy} zasygnalizowano, że w pewnych niewielkich odchyleniach zależność ta może być opisana wzorem $\theta = \frac{L}{d_k} \alpha$.

Model symulacyjny \textsc{SimMechanics} eliminuje ten problem, gdyż pozycja i prędkość wału belki odczytywane są bezpośrednio ze złącza obrotowego, w którym osadzony jest wspomniany wał. Niemniej jednak problem nadal występuje w rzeczywistym układzie sterowania, dlatego postanowiono zmierzyć zależność $\theta(\alpha)$ poprzez symulację zbudowanego modelu. W tym celu podano stałe sterowanie na silnik i zmierzono jednocześnie kąt obrotu wału silnika i wału belki.

\begin{figure}[h]
    \centering
    \includesvg[width=\textwidth,svgpath=./vector_graphics/]{zaleznosc_kata}
    \caption{Zależność kąta obrotu wału belki od kąta obrotu wału silnika.}
    \label{fig:zaleznosc_kata_belki_od_enkodera}
\end{figure}

Należy zauważyć, że brak dobranych poprawnie parametrów silnika lub kulki nie wpływa negatywnie na dokładność aproksymacji, gdyż elementy łączące belkę z silnikiem są sztywne, a pomiar nie dotyczył dynamiki (prędkość, przyspieszenie) lecz statyki (obrót).

Obrót belki w funkcji obrotu wału silnika postanowiono aproksymować sinusoidą (\cref{fig:zaleznosc_kata_belki_od_enkodera}) o następującym wzorze:

\begin{equation}\label{eq:aproksymacja_kata_walu}
    \theta(\alpha) = A \sin (\omega \alpha + \phi) + B
\end{equation}
gdzie:
\begin{itemize}
    \item $A$ --- amplituda sinusoidy,
    \item $\omega$ --- pulsacja,
    \item $\phi$ --- przesunięcie fazowe,
    \item $B$ --- wyraz wolny.
\end{itemize}

Po przeprowadzeniu aproksymacji numerycznej za pomocą funkcji \texttt{fminsearch} z programu \textsc{Matlab} otrzymano następujące wartości parametrów funkcji \eqref{eq:aproksymacja_kata_walu}: $A = \num{0.1181}$, $\omega = \num{0.9963}$, $\phi = \num{-1.7076}$, $B = \num{-0.0099}$; przybliżenie zostało przedstawione na \cref{fig:zaleznosc_kata_belki_od_enkodera}. Taka funkcja nie jest trudna do implementacji na sterowniku PLC, nie powinna również stanowić zbyt dużego wyzwania obliczeniowego\footnote{Producent deklaruje min. \SI{18}{\micro\second} dla obliczeń zmiennoprzecinkowych.} dla wybranego procesora (zob. tabelę \ref{tab:parametry_PLC_SB}).

\section{Podsumowanie}

W niniejszym rozdziale opisano sposób budowy modeli za pomocą przybornika narzędziowego \textsc{Simscape Multibody}, następnie przedstawiono i omówiono zbudowany w ten sposób model symulacyjny obiektu regulacji. W kolejnej części opisano model matematyczny silnika prądu stałego i jego implementację. Na końcu zaprezentowano wykorzystanie modelu obiektu do znalezienia zależności geometrycznej między kątem obrotu wału silnika i wału belki.

%---------------------------------------------------------------------------
\chapter{Identyfikacja}
\label{cha:ch5_identyfikacja}

Lorem ipsum.

%---------------------------------------------------------------------------
\chapter{Model liniowy obiektu i regulacja}
\label{cha:ch6_model_liniowy}

Do wyznaczenia regulatorów dla systemu konieczne jest uzyskanie jego uproszczonego modelu liniowego (\cite{TEORIASTER}), który --- w myśl twierdzenia Hartmana-Grobmana --- zachowuje się podobnie jak obiekt nieliniowy. Zgodnie z teorią sterowania, wyznaczone w tym rozdziale regulatory powinny działać w odchyłkach wybranego punktu pracy.

%%%%
\section{Punkt pracy i linearyzacja}
\label{sec:ch6_punkt_pracy_linearyzacja}

Za zmienne stanu przyjęto: położenie liniowe kulki względem środka belki $x_1$, kąt obrotu belki względem poziomu $x_3$ oraz ich pochodne w czasie $x_2 = \dot{x_1}$, $x_4 = \dot{x_3}$.

\begin{figure}[h]
    \centering
    \includesvg[width=0.5\textwidth,svgpath=./vector_graphics/]{zmienne_stanu}
    \caption{Reprezentacja głównych zmiennych stanu w obiekcie.}
    \label{fig:zmienne_stanu}
\end{figure}

Jak naturalny wybrano punkt pracy $x^*$, w którym wszystkie zmienne stanu są zerowe:
\begin{equation}
    x^* = \begin{bmatrix}
    0 \\
    0 \\
    0 \\
    0 \\
    \end{bmatrix}
\end{equation}

Linearyzację modelu przeprowadzono z wykorzystaniem narzędzi wbudowanych w pakiet \textsc{Mat\-lab/Si\-mu\-link}, a konkretnie \texttt{Linear Analysis Tool}. Do linearyzacji przygotowano wersję mo\-de\-lu układu z odłączonym wejściem na zakłócenia oraz wyłączoną (za pomocą bloku \textit{Manual Switch}) częścią odpowiedzialną za symulację tarcia suchego w silniku (\cref{fig:sm_linearization}). Usunięcie tarcia suchego w~przypadku linearyzacji w punkcie równowagi $x^*$ spowodowane jest względami praktycznymi: częsta zmiana znaku prędkości obrotowej belki utrudnia lub nawet uniemożliwia uruchamianie symulacji, gdyż wewnętrzny mechanizm \textsc{Simulink} wykrywa wtedy wiele przejść przez zero\footnote{Możliwe jest wyłączenie tego mechanizmu lub użycie metody adaptacyjnej wykrywania przejść przez zero, jednakże spowalnia to symulację.}.

\begin{figure}[h]
    \centering
    \includegraphics[width=0.9\textwidth]{sm_linearization}
    \caption{Model układu przeznaczony do przeprowadzenia linearyzacji.}
    \label{fig:sm_linearization}
\end{figure}

Przed linearyzacją konieczne było znalezienie punktu pracy, czyli przeprowadzenie tzw. \textit{trimmingu} modelu. W procesie tym wyznacza się ograniczenia na stany układu lub ich wartości początkowe. Jest to istotne zagadnienie, gdyż układ zbudowany za pomocą przybornika \textsc{Simscape Multibody} zawiera \num{18} stanów, chociaż, tak jak to zdefiniowano na początku rozdziału, wybrano jedynie \num{4} do prezentacji systemu. Większość z tych ,,nadmiarowych'' stanów to położenia i prędkości złącz pryzmatycznych lub obrotowych, które nie są istotne z punktu widzenia reprezentacji liniowej układu.

\textit{Trimming} modelu to tak naprawdę proces optymalizacyjny, mający na celu tak dobrać wartości początkowe stanów systemu, aby maksymalny ich błąd w trakcie symulacji był jak najmniejszy; jest to jednoznaczne z numerycznym znajdowaniem punktu równowagi systemu. Z powodu swojej optymalizacyjnej natury, proces ten nie mógł znaleźć odpowiedniego punktu, dopóki nie został rozpoczęty z odpowiednio bliskiego otoczenia punktu równowagi. Osiągnięto to podając jako wartość startową dla wejścia bloku (\textit{voltage} na \cref{fig:sm_linearization}) napięcie sterowania $u = -0,51158$. Wartość ta wynika z faktu delikatnego ciążenia belki w stronę mechanizmu korbowego, co musi rekompensować silnik poprzez wytworzenie niezerowego momentu.

Po wyznaczeniu punktu równowagi, nadal wykorzystując narzędzie \texttt{Linear Analysis Tool}, zlinearyzowano układ otrzymując wynikowe macierze:

\begin{nalign}
    \widetilde{A} &= \begin{bmatrix}
        0 & 1 & 0 & 0 \\
        -0,2289 & 0 & 47,607 & 3,2117 \\
        0 & 0 & 0 & 1 \\
        0,4977 & 0 & 0,262 & -6,9832 \\
    \end{bmatrix} \\
    \widetilde{B} &= \begin{bmatrix}
        0 \\
        -15,5016 \\
        0 \\
        33,7051 \\
    \end{bmatrix} \\
    \widetilde{C} &= \begin{bmatrix}
            0,03 & 0 & 0 & 0 \\
            0 & 0,03 & 0 & 0 \\
            0 & 0 & 0,2044 & 0 \\
            0 & 0 & 0 & 0,2044 \\
    \end{bmatrix} \\
    \widetilde{D} &= 0 \label{eq:macierze_stanu1}
\end{nalign}

Dalsza analiza wyników linearyzacji wykazała, że narzędzie \texttt{Linear Analysis Tool} dobrało inne zmienne stanu niż oczekiwane, a mianowicie:

\begin{itemize}
    \item $\tilde{x}_1$ -- kąt obrotu kulki wokół własnej osi,
    \item $\tilde{x}_2$ -- prędkość kątowa obrotu kulki wokół własnej osi,
    \item $\tilde{x}_3$ -- kąt obrotu wału motoreduktora,
    \item $\tilde{x}_4$ -- prędkość kątowa obrotu wału motoreduktora.
\end{itemize}

Należy zauważyć, że kąt obrotu kulki wokół własnej osi $\tilde{x}_1$ oraz przesunięcie liniowe kulki $x_1$ są ze sobą liniowo zależne: $\tilde{x}_1 = r x_1$, gdzie $r$ to promień kulki. Podobnie kąty obrotu belki $x_3$ i obrotu wału motoreduktora $\tilde{x}_3$ również są od siebie liniowo zależne w pobliżu punktu pracy, co zostało wskazane w rozdziale \ref{sec:ch3_uklad_napedowy}, wzór \eqref{eq:uproszczona_zaleznosc_kata_belki}. Oznacza to, że można dokonać transformacji macierzy stanu \eqref{eq:macierze_stanu1} tak, aby otrzymać oczekiwane zmienne stanu.

Transformację podobieństwa zmiennych stanu $\breve{x} = T_1 \tilde{x}$ przeprowadza się następująco:
\begin{nalign}
    \breve{x}' &= T_1 \widetilde{A} T_1^{-1} \breve{x} + T_1 \widetilde{B} u \\
    \breve{y} &= \widetilde{C} T_1^{-1} \breve{x} + \widetilde{D} u \label{eq:transformacja_x}
\end{nalign}

Jako macierz transformacji $T_1$ użyto macierzy wyjścia $\widetilde{C}$ i otrzymano następujące macierze wynikowe:
\begin{nalign}
    \breve{A} &= \begin{bmatrix}
         0 & 1 & 0 & 0 \\
         -0,2289 & 0 & 6,9871 & 0,4714 \\
         0 & 0 & 0 & 1 \\
         3,3914 & 0 & 0,262 & -6,9832 \\
    \end{bmatrix} \\
    \breve{B} &= \begin{bmatrix}
         0 \\
         -0,465 \\
         0 \\
         6,8896 \\
    \end{bmatrix} \\
    \breve{C} &= \begin{bmatrix}
        1 & 0 & 0 & 0 \\
        0 & 1 & 0 & 0 \\
        0 & 0 & 1 & 0 \\
        0 & 0 & 0 & 1 \\
    \end{bmatrix} \\
    \breve{D} &= 0 \label{eq:macierze_stanu2}
\end{nalign}

% TODO: lepsza argumentacja?
Jak można zaobserwować, położenie zer i jedynek w macierzach $\widetilde{A}$ i $\breve{A}$ pozostało takie samo, co potwierdza, że dokonana została transformacja liniowa.

%%%%
\section{Kaskadowy układ regulacji}
\label{sec:ch6_kaskadowy_uklad_regulacji}

Należy zauważyć, że belka wraz z mechanizmem napędowym tworzą szybki, nieliniowy, niejednoznaczny i łatwo zakłócany układ, podczas gdy kulka jest obiektem wolnym, liniowym i stacjonarnym. Ta ,,rozdzielność'' dwóch układów sugeruje możliwość zastosowania dwóch regulatorów.

Ponadto w tym obiekcie regulacji zastosowano silnik prądu stałego zamiast rozwiązania z wbudowanym regulatorem pozycji, na przykład serwonapędem modelarskim (porównanie alternatywnych do zastosowanego wariantów zespołu napędowego umieszczono w dodatku \ref{appA_warianty_zespolu_napedowego}). W wyniku dzia\-łania tego silnika następuje ruch belki, który powoduje ruch kulki. Masa kulki jest niewielka, a zatem nie wpływa praktycznie wcale na ruch belki.

% TODO: czas teraźniejszy?!
Korzystając z powyższych obserwacji i wspomnianego ciągu następstw, w niniejszej pracy proponuje się zastosowanie kaskadowego układu regulacji, w którym regulator nadrzędny pozycji kulki generuje sygnał sterujący dla regulatora podrzędnego wychylenia belki. Schemat ilustrujący taką strukturę sterowania przedstawiony jest na \cref{fig:kaskadowy_uklad_regulacji}.

% TODO: wymień na Tikz
\begin{figure}[h]
    \centering
    \includegraphics[width=1\textwidth]{kaskadowy_uklad_regulacji}
    \caption{Schemat ilustrujący ideę kaskadowego układu regulacji obiektu typu kulka i~belka.}
    \label{fig:kaskadowy_uklad_regulacji}
\end{figure}

\pagebreak

Aby możliwe było obliczenie osobnych regulatorów dla kulki oraz dla belki, równanie stanu powinno mieć następującą strukturę:

\begin{equation}
    \renewcommand\arraystretch{2}
    \begin{matrix}
        \text{kulka}~\Bigg\{ \\
        \text{belka}~\Bigg\{ 
    \end{matrix}
    \renewcommand\arraystretch{1}
    \left[
    \begin{array}{c}
        \bar{x}_1' \\
        \bar{x}_2' \\
        \hline
        \bar{x}_3' \\
        \bar{x}_4'
    \end{array}
    \right]
    =
    \underbrace{
        \renewcommand\arraystretch{2}
        \left[
        \begin{array}{c|c}
           \bar{A}_{11} & \bar{A}_{12} \\
           \hline
           0 & \bar{A}_{22}
        \end{array}
        \right]
    }_{\bar{A}}
    \left[
    \begin{array}{c}
        \bar{x}_1 \\
        \bar{x}_2 \\
        \hline
        \bar{x}_3 \\
        \bar{x}_4 \\
    \end{array}
    \right]
    +
    \underbrace{
        \renewcommand\arraystretch{2}
        \left[
        \begin{array}{c}
        0 \\
        \hline
        \bar{B}_2
        \end{array}
        \right]
    }_{\bar{B}}
    u \label{eq:warunki_kaskady}
\end{equation}

Z \eqref{eq:warunki_kaskady} wynika, że zachowanie kulki zależy od kulki i belki, a nie zależy od sterowania, natomiast zachowanie belki zależy tylko od belki i sterowania:

\begin{align*}
    \begin{bmatrix}
        \bar{x}_1' \\ \bar{x}_2'
    \end{bmatrix}
    &= \begin{bmatrix}
    \bar{A}_{11} & \bar{A}_{12}
    \end{bmatrix} \bar{x} \\
    \begin{bmatrix}
    \bar{x}_3' \\ \bar{x}_4'
    \end{bmatrix}
    &= \begin{bmatrix}
    0 & \bar{A}_{22}
    \end{bmatrix} \bar{x}
    + \bar{B}_2 u
\end{align*}

Jak można łatwo zauważyć, struktury macierzy $\breve{A}$ i $\breve{B}$ \eqref{eq:macierze_stanu2} nie odpowiadają strukturom pożądanych macierzy $\bar{A}$ i $\bar{B}$. Może to być spowodowane bezwładnością kulki: w momencie ruchu belki w dół, kulka pod wpływem tarcia przesuwa się odrobinę względem powierzchni belki w stronę przeciwną do kierunku spadku (ilustracja na \cref{fig:przeciwny_ruch_kulki}). Sugeruje to niewielka i ujemna zależność przyspieszenia kątowego kulki $\breve{x}_2'$ od sterowania~$u$, podczas gdy zależność belki od sterowania ma znak dodatni i dużo większy współczynnik.

\begin{figure}[h]
    \centering
    \includesvg[width=0.6\textwidth,svgpath=./vector_graphics/]{przeciwny_ruch_kulki}
    \caption{Ilustracja toczenia się kulki w przeciwną stroną niż kierunek spadku.}
    \label{fig:przeciwny_ruch_kulki}
\end{figure}

Kompensacji takiego ruchu kulki, a więc doprowadzenia macierzy $\breve{B}$ do struktury macierzy $\bar{B}$, można dokonać przez transformację $\hat{x} = T_2 \breve{x}$ zdefiniowaną jak w \eqref{eq:transformacja_x}; za macierz $T_2$ dobrano:
\begin{equation}\label{eq:transformacja2_x}
    \renewcommand\arraystretch{1.2}
    T_2 = \begin{bmatrix}
        1 & 0 & -\frac{\breve{B}_2}{\breve{B}_4} & 0 \\
        0 & 1 & 0 & -\frac{\breve{B}_2}{\breve{B}_4} \\
        0 & 0 & 1 & 0 \\
        0 & 0 & 0 & 1 \\
    \end{bmatrix}
\end{equation}
gdzie $\breve{B}_i$ oznacza \textit{i}-ty element macierzy $\breve{B}$. Współczynnik $p=-\frac{\breve{B}_2}{\breve{B}_4}$ został dobrany w taki sposób, aby wyzerować element drugi macierzy $\breve{B}$ przy użyciu macierzy transformacji, która jak najmniej ingeruje w stan systemu. Stąd obecność jedynek na przekątnej $T_2$, natomiast współczynnik $p$ jest rozwiązaniem równania $\hat{B} = T_2 \breve{B}$ dla jej drugiego wiersza:
\begin{equation*}
0 =
    \begin{bmatrix}
        0 & 1 & 0 & p
    \end{bmatrix} \cdot
    \begin{bmatrix}
        0 \\ \breve{B}_2 \\ 0 \\ \breve{B}_4
    \end{bmatrix}
  = \breve{B}_2 + p \breve{B}_4
\end{equation*}
skąd już dzięki prostemu przekształceniu otrzymujemy $p = -\frac{\breve{B}_2}{\breve{B}_4}$.

Współczynnik $p$ pojawia się również w pierwszym wierszu macierzy transformacji $T_2$. Został on umieszczony tam w celu uzyskania zgodności zależności zmiennych stanu  $\hat{x}_1$ od $\hat{x}_2$ tak, jak zależne są $\breve{x}_1$ od $\breve{x}_2$.

Dzięki przekształceniu przez macierz \eqref{eq:transformacja2_x} otrzymano następujące macierze stanu:
\begin{nalign}
    \hat{A} &= \begin{bmatrix}
    0 & 1 & 0 & 0 \\
    0 & 0 & 7,0047 & 0 \\
    0 & 0 & 0 & 1 \\
    3,3914 & 0 & 0,0331 & -6,9832 \\
    \end{bmatrix}
    =
    \renewcommand\arraystretch{2}
    \left[
        \begin{array}{c|c}
        \hat{A}_{11} & \hat{A}_{12} \\
        \hline
        \hat{A}_{21} & \hat{A}_{22}
        \end{array}
    \right] \\
    \hat{B} &= \begin{bmatrix}
    0 \\
    0 \\
    0 \\
    6,8896 \\
    \end{bmatrix} \\
    \hat{C} &= \begin{bmatrix}
    1 & 0 & -0,0675 & 0 \\
    0 & 1 & 0 & -0,0675 \\
    0 & 0 & 1 & 0 \\
    0 & 0 & 0 & 1 \\
    \end{bmatrix} \\
    \hat{D} &= 0 \label{eq:macierze_stanu3}
\end{nalign}

Widać, że postać macierzy $\hat{B}$ z \eqref{eq:macierze_stanu3} odpowiada już oczekiwanej strukturze, natomiast macierz $\hat{A}$~jeszcze nie ma takiej formy. W związku z tym zdecydowano się doprowadzić macierz $\hat{A}$ do postaci blokowo-trójkątnej górnej poprzez wyzerowanie $\hat{A}_{21}$. Nowe macierze stanu przedstawiają się następująco:
\begin{nalign}
    A &= \begin{bmatrix}
    0 & 1 & 0 & 0 \\
    0 & 0 & 7,0047 & 0 \\
    0 & 0 & 0 & 1 \\
    0 & 0 & 0,0331 & -6,9832 \\
    \end{bmatrix} \\
    B &= \begin{bmatrix}
    0 \\
    0 \\
    0 \\
    6,8896 \\
    \end{bmatrix} \\
    C &= \begin{bmatrix}
    1 & 0 & -0,0675 & 0 \\
    0 & 1 & 0 & -0,0675 \\
    0 & 0 & 1 & 0 \\
    0 & 0 & 0 & 1 \\
    \end{bmatrix} \\
    D &= 0 \label{eq:macierze_stanu4}
\end{nalign}

Aby udowodnić, że wyzerowanie ww. elementu macierzy $\hat{A}$ nie przyniosło negatywnych skutków, poniżej przedstawiono wykresy charakterystyk amplitudowo-fazowych dla systemów opisanych macierzami $\hat{A}$, $\hat{B}$, $\hat{C}$ i $\hat{D}$ oraz $A$, $B$, $C$ i $D$.

Jak widać na \cref{fig:charakterystyka_amplitudowo_fazowa}, w bardzo niskich częstotliwościach (poniżej \SI{1}{\hertz}) układy nie zachowują się jednakowo. Wynika to z faktu, że przy tak powolnych ruchach belką kulka jest w stanie potoczyć się dalej od środka belki, a zatem jej oddziaływanie na belkę rośnie. Jednakże w wyższych częstotliwościach charakterystyki układów są takie same i w związku z tym można założyć, że macierze wynikowe $A$, $B$, $C$ i $D$ są poprawne.

Ostatecznie uzyskano następujące równania stanu:
\begin{align}
\begin{bmatrix}
    \dot{x}_1 \\ \dot{x}_2 \\ \dot{x}_3 \\ \dot{x}_4
\end{bmatrix}
&= \begin{bmatrix}
    0 & 1 & 0 & 0 \\
    0 & 0 & 7,0047 & 0 \\
    0 & 0 & 0 & 1 \\
    0 & 0 & 0,0331 & -6,9832
\end{bmatrix}
\begin{bmatrix}
    x_1 \\ x_2 \\ x_3 \\ x_4
\end{bmatrix}
+
\begin{bmatrix}
    0 \\ 0 \\ 0 \\ 6,8896
\end{bmatrix}
u \label{eq:rownania_stanu} \\%\displaybreak\\
\begin{bmatrix}
    y_1 \\ y_2 \\ y_3 \\ y_4
\end{bmatrix}
&= \begin{bmatrix}
    1 & 0 & -0,0675 & 0 \\
    0 & 1 & 0 & -0,0675 \\
    0 & 0 & 1 & 0 \\
    0 & 0 & 0 & 1 \\
\end{bmatrix}
\begin{bmatrix}
x_1 \\ x_2 \\ x_3 \\ x_4
\end{bmatrix} \label{eq:rownania_wyjscia} 
\end{align}

Za pomocą kryterium macierzy Kalmana stwierdzono, że układ \eqref{eq:rownania_stanu} jest sterowalny, a zatem możliwe jest dobranie regulatora, który ten układ doprowadzi do dowolnego punktu pracy. Ponadto macierz stanu układu posiada następujące wartości własne: $\begin{bmatrix}
0 & 0 & 0.0047 & -6.9880
\end{bmatrix}^\intercal$, co jest sprzeczne z założeniami twierdzenia Hartmana-Grobmana (\cite{TEORIASTERCW}), a zatem nie pozwala twierdzić o podobieństwie reprezentacji systemu zlinearyzowanego do systemu nieliniowego.

\begin{figure}[ht]
    \centering
    \includesvg[width=1\textwidth,svgpath=./vector_graphics/]{ch_amplitudowo_fazowa}
    \caption{Wykresy amplitudowo-fazowe systemów opisanych macierzami \eqref{eq:macierze_stanu3} oraz \eqref{eq:macierze_stanu4}. Przedstawiona charakterystyka dla pierwszej zmiennej stanu (położenia linio\-wego kulki).}
    \label{fig:charakterystyka_amplitudowo_fazowa}
\end{figure}

Do uzyskania struktury kaskadowej systemu zlinearyzowanego należy wydzielić równania stanu kulki oraz belki:

\begin{align}
\begin{bmatrix}
\dot{x}_1 \\ \dot{x}_2
\end{bmatrix}
&= \begin{bmatrix}
    0 & 1 & 0 \\
    0 & 0 & 7,0047
\end{bmatrix}
\begin{bmatrix}
    x_1 \\ x_2 \\ x_3
\end{bmatrix} \label{eq:rownania_stanu_kulki} \\
\begin{bmatrix}
    y_1 \\ y_2
\end{bmatrix}
&= \begin{bmatrix}
    1 & 0 & -0,0675 & 0 \\
    0 & 1 & 0 & -0.0675 \\
\end{bmatrix}
\begin{bmatrix}
x_1 \\ x_2 \\ x_3 \\ x_4
\end{bmatrix} \label{eq:rownania_wyjscia_kulki} \\
\begin{bmatrix}
    \dot{x}_3 \\ \dot{x}_4
\end{bmatrix}
&= \begin{bmatrix}
    0 & 1 \\
    0,0331 & -6,9832
\end{bmatrix}
\begin{bmatrix}
    x_3 \\ x_4
\end{bmatrix}
+
\begin{bmatrix}
    0 \\ 6,8896
\end{bmatrix}
u \label{eq:rownania_stanu_belki} \\%\displaybreak\\
\begin{bmatrix}
    y_3 \\ y_4
\end{bmatrix}
&= \begin{bmatrix}
    1 & 0 \\
    0 & 1 \\
\end{bmatrix}
\begin{bmatrix}
    x_3 \\ x_4
\end{bmatrix} \label{eq:rownania_wyjscia_belki} 
\end{align}

Oczywiście taka forma równań \eqref{eq:rownania_stanu_kulki} i \eqref{eq:rownania_wyjscia_kulki} jest niepoprawna, ale bardzo łatwo można ją przekształcić do schematu z \cref{fig:kaskadowy_uklad_regulacji} stosując podmianę $u_1^* = x_3$, $u_2^* = x_4$: %(równania \eqref{eq:rownania_stanu_kulki2} oraz \eqref{eq:rownania_wyjscia_kulki2}). 
\begin{align}
\begin{bmatrix}
    \dot{x}_1 \\ \dot{x}_2
\end{bmatrix}
&= \begin{bmatrix}
    0 & 1 \\
    0 & 0
\end{bmatrix}
\begin{bmatrix}
x_1 \\ x_2
\end{bmatrix}
+
\begin{bmatrix}
    0 & 0 \\ 7,0047 & 0
\end{bmatrix}
\begin{bmatrix}
    u_1^* \\ u_2^*
\end{bmatrix} \label{eq:rownania_stanu_kulki2} \\
\begin{bmatrix}
    y_1 \\ y_2
\end{bmatrix}
&= \begin{bmatrix}
    1 & 0 \\
    0 & 1 \\
\end{bmatrix}
\begin{bmatrix}
    x_1 \\ x_2
\end{bmatrix}
+ \begin{bmatrix}
    -0,0675 & 0 \\
    0 & -0.0675
\end{bmatrix}
\begin{bmatrix}
    u_1^* \\ u_2^*
\end{bmatrix} \label{eq:rownania_wyjscia_kulki2}
\end{align}

%%%%
\section{Regulator pochylenia belki}
\label{sec:ch6_regulator_belki}

Jako regulator pochylenia belki wybrano prosty regulator proporcjonalny od stanu. Struktura regulacji przedstawiona jest na \cref{fig:schemat_regulacji_belka}, gdzie:
\begin{itemize}
    \item $\alpha_\text{ref}$ oraz $\omega_\text{ref}$ są wartościami zadanymi odpowiednio kąta oraz prędkości kątowej belki,
    \item $K_b$ jest macierzą wzmocnień regulatora,
    \item $G_b(s)$ jest transmitancją (typu \textit{SIMO}) zlinearyzowanego systemu belki opartego o równania stanu \eqref{eq:rownania_stanu_belki} i wyjścia \eqref{eq:rownania_wyjscia_belki},
    \item $y_3$, $y_4$ są wartościami wyjściowymi z modelu (pozycją kątową i prędkością kątową belki).
\end{itemize}

\begin{figure}[ht]
    \centering
    
    \begin{tikzpicture}[auto, node distance=1cm,>=latex']
        \node [input, name=input] {};
        \node [sum, right=2of input] (sum) {};
        \node [gain, right=of sum] (controller) {$K_b$};
        \node [block, right=of controller] (plant) {$G_b(s)$};
        \node [output, right=of plant] (output) {};
        \draw [draw,->] (input) -- node {$\left[\alpha_\text{ref}, ~\omega_\text{ref}\right]^\intercal$} (sum);
        \draw [->] (sum) -- node {} (controller);
        \draw [->] (controller) -- node {} (plant);
        \draw [->] (plant) -- node [name=y] {$\left[y_3, ~y_4\right]^\intercal$}(output);
        \draw [->] (y) -- ++ (0,-1.5) -| node [pos=0.95] {$-$} (sum);
    \end{tikzpicture}
    
    \caption{Schemat blokowy sterowania zlinearyzowanego systemu odpowiadającego za zachowanie belki.}
    \label{fig:schemat_regulacji_belka}
\end{figure}

Wartość wzmocnienia $K_b$ została dobrana za pomocą funkcji \texttt{looptune} z programu \textsc{Matlab}, co pozwoliło uzyskać dużą dynamikę działania regulatora. Wspomniana funkcja optymalizuje pętlę sprzężenia zwrotnego układu zaprezentowaną na \cref{fig:schemat_regulacji_looptune} według ustalonych kryteriów częstotliwościowych: \begin{itemize}
    \item pasmo przenoszenia -- punkt przejścia charakterystyki amplitudy przez zero musi zawierać się w~wyznaczonym zakresie częstotliwości,
    \item wydajność -- akcja całkująca na częstotliwościach poniżej wyznaczonego zakresu,
    \item odporność (ang. \textit{robustness}) -- odpowiednie zapasy stabilności na częstotliwościach poniżej wyznaczonego zakresu.
\end{itemize}

Do funkcji \texttt{looptune} podano jako argument zakres częstotliwości \SIrange{5}{20}{\hertz}, uzyskując wzmocnienie regulatora $K_b = \left[-8,2069;~  -1,5756\right]$. Informacje o sposobie implementacji regulatora zamieszczone są w rozdziale \ref{subsec:ch7_regulator_belki}.

\begin{figure}[ht]
    \centering
    
    \begin{tikzpicture}[auto, node distance=1cm,>=latex']
    \node [block] (plant) {$G(s)$};
    \node [block, below=of plant] (controller) {$C(s)$};
    \draw [->] (plant.east) -- ++ (1,0) |- node {$y$} (controller.east);
    \draw [->] (controller.west) -- ++ (-1,0) |- node {$u$} (plant.west);
    %\draw [->] (y) -- ++ (0,-1.5) -| node [pos=0.95] {$-$} (sum);
    \end{tikzpicture}
    
    \caption{Schemat blokowy układu optymalizowanego przez funkcję \texttt{looptune}; $G(s)$ to obiekt regulacji, $C(s)$ to regulator.}
    \label{fig:schemat_regulacji_looptune}
\end{figure}

% TODO: charakterystyka bodego? :(

%%%%
\section{Regulator pozycji kulki}
\label{sec:ch6_regulator_kulki}

Po uzyskaniu odpowiedniej struktury wewnętrznej pętli sprzężenia zwrotnego (\cref{fig:schemat_regulacji_belka}), w podobny sposób zaprojektowano zewnętrzną pętlę sprzężenia zwrotnego, obejmującą swym działaniem oba systemy zlinearyzowane -- kulkę oraz belkę. Schemat zbudowanego systemu przedstawiono na \cref{fig:schemat_regulacji_belka_i_kulka}; system korzysta z wyprowadzonej w równaniach \eqref{eq:rownania_stanu_kulki2} oraz \eqref{eq:rownania_wyjscia_kulki2} postaci równań stanu systemu kulki.

\begin{figure}[ht]
    \centering
    
    \begin{tikzpicture}[auto, node distance=1cm,>=latex']
        \node [input, name=input] {};
        \node [sum, right=1.5of input] (sum_outer) {};
        \node [gain, right=of sum_outer] (K_ball) {$K_k$};
        \node [sum, right=1.5of K_ball] (sum_inner) {};
        \node [gain, right=of sum_inner, minimum width=1cm] (K_beam) {$K_b$};
        \node [block, right=of K_beam] (beam) {$G_b(s)$};
        \node [block, right=1.5of beam] (ball) {$G_k(s)$};
        \node [output, right=of ball] (output) {};
        \node [mux, below=0.5of beam] (mux) {};
        \draw [draw,->] (input) -- node {$\begin{bmatrix}s_\text{ref} \\ v_\text{ref} \\ \alpha_\text{ref}^* \\ \omega_\text{ref}^*\end{bmatrix}$} (sum_outer);
        \draw [->] (sum_outer) -- node {} (K_ball);
        \draw [->] (K_ball) -- node {$\begin{bmatrix}\alpha_\text{ref} \\ \omega_\text{ref}\end{bmatrix}$} (sum_inner);
        \draw [->] (sum_inner) -- node {} (K_beam);
        \draw [->] (K_beam) -- node {} (beam);
        \draw [->] (beam) -- node [name=Y_beam] {$\begin{bmatrix}y_3 \\ y_4\end{bmatrix}$} (ball);
        \draw [->] (ball) -- node [name=Y_ball] {$\begin{bmatrix}y_1 \\ y_2\end{bmatrix}$} (output);
        \draw [->] (Y_beam) -- ++ (0,-1.5) -| node [pos=0.93] {$-$} (sum_inner);
        \draw [->] (Y_beam) |- (mux.300);
        \draw [->] (Y_ball) |- (mux.60);
        \draw [->] (mux) -| node [pos=0.95] {$-$} (sum_outer);
    \end{tikzpicture}
    
    \caption{Schemat sterowania zlinearyzowanego systemu odpowiadającego za zachowanie belki oraz kulki.}
    \label{fig:schemat_regulacji_belka_i_kulka}
\end{figure}

Na \cref{fig:schemat_regulacji_belka_i_kulka} zastosowano następujące oznaczenia:
\begin{itemize}
    \item $s_\text{ref}$, $v_\text{ref}$, $\alpha_\text{ref}^*$, $\omega_\text{ref}^*$ to wartości zadane położenia i prędkości liniowej kulki, a także kąta i prędkości belki dla zewnętrznej pętli sprzężenia zwrotnego,
    \item $K_k$ to macierz wzmocnień regulatora nadrzędnego,
    \item $\alpha_\text{ref}$, $\omega_\text{ref}$ to wartości zadane kąta i prędkości kątowej belki wygenerowane przez regulator $K_k$,
    \item $K_b$, $G_b(s)$, $y_3$ oraz $y_4$ jak na \cref{fig:schemat_regulacji_belka},
    \item $G_k(s)$ to transmitancja (typu \textit{MIMO}) zlinearyzowanego systemu kulki opartego o równania stanu \eqref{eq:rownania_stanu_kulki2} i wyjścia \eqref{eq:rownania_wyjscia_kulki2},
    \item $y_1$, $y_2$ to wartości wyjściowe z modelu kulki (położenie i prędkość).
\end{itemize}

Względem schematu ideowego z \cref{fig:kaskadowy_uklad_regulacji} nastąpiła zmiana wektora wartości zadanych dla zewnętrznej pętli, która również została rozszerzona o dodatkowe dwa stany związane z pochyleniem i prędkością kątową belki. Zmiany te wynikają z większych możliwości kształtowania sygnału sterowania dla wewnętrznej pętli, gdy do dyspozycji regulatora $K_k$ jest więcej stanów.

Wewnętrzną pętlę sprzężenia zwrotnego oraz system $G_k(s)$ ,,zwinięto'' za pomocą komend \texttt{sumblk} oraz \texttt{connect} z programu \textsc{Matlab} w jeden system typu \textit{State-Space}. Pozwoliło to uzyskać strukturę potrzebną do użycia, podobnie jak w przypadku regulatora od stanu belki, funkcji \texttt{looptune} do wyznaczenia macierzy wzmocnień regulatora.

Funkcja \texttt{looptune} w przypadku zewnętrznej pętli sprzężenia zwrotnego wygenerowała następującą macierz wzmocnień regulatora $K_k$:
\begin{equation}
    K_k = \begin{bmatrix}
    1,7905 &  1,7813 &  0,9181 &  1,1457 \\
    0,1802 & -0,4902 &  0,1762 &  0,2744
    \end{bmatrix} \label{eq:regulator_kulki}
\end{equation}

Należy przypomnieć, że funkcja \texttt{looptune} jest funkcją optymalizacyjną; w związku z tym, jej wynik może być zależny od punktu rozpoczęcia optymalizacji. Macierz \eqref{eq:regulator_kulki} została wygenerowana z~punktu początkowego $K_k^* = \begin{bmatrix}
1 & 0 & 0 & 0 \\ 0 & 1 & 0 & 0
\end{bmatrix}$.

Po ponownym ,,zwinięciu'' regulatora $K_k$ i wewnętrznej pętli sprzężenia zwrotnego za pomocą komend \texttt{sumblk} oraz \texttt{connect} uzyskano system \textit{State-Space} całego zlinearyzowanego układu wraz z~pętlami sprzężeń zwrotnych. Macierze stanu tego systemu zostały zaprezentowane na poniżej \eqref{eq:macierze_stanu_calego_ukladu}:
\begin{nalign}
    A_c &= \begin{bmatrix}
        0 & 1 & 0 & 0 \\
        -103,4 & -79,16 & -103,2 & -95,39 \\
        0 & 0 & 0 & 1 \\
        7,005 & 0 & 0 & 0
    \end{bmatrix}  \\
    B_c &= \begin{bmatrix}
        0 & 0 & 0 & 0 \\
        103,2 & 95,39 & 53,82 & 67,76 \\
        0 & 0 & 0 & 0 \\
        0 & 0 & 0 & 0
    \end{bmatrix}  \\
    C_c &= \begin{bmatrix}
        -0,0675 & 0 & 1 & 0 \\
        0 & -0,0675 & 0 & 1 \\
        1 & 0 & 0 & 0 \\
        0 & 1 & 0 & 0 \\
    \end{bmatrix}  \\
    D_c &= \begin{bmatrix}
        0 & 0 & 0 & 0 \\
        0 & 0 & 0 & 0 \\
        0 & 0 & 0 & 0 \\
        0 & 0 & 0 & 0
    \end{bmatrix} \label{eq:macierze_stanu_calego_ukladu}
\end{nalign}

Sprawdzenie wartości własnych macierzy $A_c$ pokazuje, że układ ten ma charakter oscylacyjnie stabilny \eqref{eq:wartosci_wlasne_A_c}:
\begin{equation}
    \Lambda(A_c) = \begin{bmatrix}
     -77,9427 \\
     -0,0527 + 2,8869i \\
     -0,0527 - 2,8869i \\
     -1,1124
    \end{bmatrix} \label{eq:wartosci_wlasne_A_c}
\end{equation}

Po implementacji regulatora, opisanej w rozdziale \ref{subsec:ch7_regulator_kulki}, okazało się, że układ nie jest w stanie w~rozsądnym czasie ustabilizować położenia kulki. Dobrano więc w sposób empiryczny inne nastawy $K_k$, które polepszyły zachowanie układu. Zmiany względem oryginalnych nastaw $K_k$ zostały wyróżnione na~\eqref{eq:regulator_kulki2}:
\begin{equation}
K_k = \begin{bmatrix}
    1,7905 &  1,7813 &  0,9181 &  \mathbf{0,1} \\
    0,1802 & -0,4902 &  0,1762 &  0,2744
\end{bmatrix} \label{eq:regulator_kulki2}
\end{equation}

Po wprowadzeniu zmienionego regulatora do całości układu, w macierzach stanu nie zaszły duże zmiany, a charakter układu pozostał oscylacyjnie stabilny:
\begin{nalign}
    A_c &= \begin{bmatrix}
        0 & 1 & 0 & 0 \\
        -103,4 & \mathbf{-78,73} & -103,2 & \mathbf{-101,8} \\
        0 & 0 & 0 & 1 \\
        7,005 & 0 & 0 & 0
    \end{bmatrix} \\
    B_c &= \begin{bmatrix}
        0 & 0 & 0 & 0 \\
        103,2 & \mathbf{101,8} & 53,82 & 67,76 \\
        0 & 0 & 0 & 0 \\
        0 & 0 & 0 & 0
    \end{bmatrix} \\
    C_c &= \begin{bmatrix}
        -0,0675 & 0 & 1 & 0 \\
        0 & -0,0675 & 0 & 1 \\
        1 & 0 & 0 & 0 \\
        0 & 1 & 0 & 0 \\
    \end{bmatrix} \\
    D_c &= \begin{bmatrix}
        0 & 0 & 0 & 0 \\
        0 & 0 & 0 & 0 \\
        0 & 0 & 0 & 0 \\
        0 & 0 & 0 & 0
    \end{bmatrix} \\
    \Lambda(A_c) &= \begin{bmatrix}
    -77.5116 \\
    -0.0844 + 2.9825i \\
    -0.0844 - 2.9825i \\
    -1.0475
    \end{bmatrix} \label{eq:macierze_stanu_calego_ukladu2}
\end{nalign}

%%%%
\section{Podsumowanie}

W niniejszym rozdziale przedstawiono obrane zmienne stanu i proces linearyzacji wykorzystujący narzędzia dostępne dla projektantów systemów sterowania w pakiecie \textsc{Matlab/Simulink}. Pokazano również transformację, której konieczne było poddanie wyniku linearyzacji, aby uzyskać pożądane zmienne stanu, a nie wybrane przez narzędzie \texttt{Linear Analysis Tool}. Następnie przedstawiono ideę i strukturę kaskadowego układu sterownia oraz powody, dla których postanowiono taką strukturę zastosować, a także przeprowadzono szereg transformacji i zmian w wynikowych macierzach stanu układu zlinearyzowanego, aby doprowadzić go do pożądanej struktury.

Na koniec w rozdziale przedstawiono dobór regulatorów podrzędnego, odpowiadającego za sterowanie silnikiem i belką, oraz nadrzędnego, kontrolującego pozycję kulki poprzez generowanie wartości zadanych dla regulatora podrzędnego.

%---------------------------------------------------------------------------
\chapter{Algorytmy sterowania}
\label{cha:ch7_algorytmy_sterowania}

Lorem ipsum.

%---------------------------------------------------------------------------
\chapter{Algorytmy samostrojenia}
\label{cha:ch8_algorytmy_samostrojenia}

W literaturze do teorii sterowania znajdują się opisy wielu klasycznych metod strojenia regulatorów, takich jak metoda Zieglera-Nicholsa \cite{KOWAL}\cite{CTRLENG} czy Åströma-Hägglunda \cite{CTRLENG}\cite{ASTROMHAGGLUND}. Wielu producentów oprogramowania serwonapędów lub sterowników technologicznych dostarcza własne algorytmy strojenia \textit{online} \cite{S7MANUAL}\cite{SCL_S71200_S71500}.

Zdecydowano się zrezygnować z metod klasycznych strojenia algorytmów, gdyż są one powszechnie znane i polegają na doprowadzeniu obiektu do cyklu granicznego, co w przypadku zastosowanej konstrukcji przeniesienia napędu można uzyskać podając stałe sterowanie na silnik. Prawdopodobnie dokładnie z takiego powodu oprogramowanie służące do programowania sterownika PLC --- \textsc{Siemens TIA Portal} --- nie było w stanie zakończyć poprawnie procesu samostrojenia regulatora pozycji belki.

Biorąc pod uwagę powyższe ograniczenia, zdecydowano się oprzeć samostrojenie regulatorów podrzędnego i nadrzędnego o automatyczne procesy identyfikacji matematycznych obiektów belki oraz kulki, co zostało opisane w niniejszym rozdziale.

%%%%
\section{Identyfikacja obiektu belki}
\label{sec:ch8_identyfikacja_belki}

Identyfikację belki oparto o odpowiedź skokową modelu obiektu pierwszego rzędu opisanego transmitancją:
\begin{equation}
    G_b(s) = \frac{K_b}{T_b s+1}
    \label{eq:transmitancja_obiektu_pierwszego_rzedu}
\end{equation}
gdzie $K_b$ oznacza wzmocnienie obiektu, a $T_b$ jego stałą czasową (\cite{KOWAL}). Parametry te można odczytać z~wykresu odpowiedzi skokowej (\cref{fig:odpowiedz_skokowa_obiektu_pierwszego_rzedu}, równanie \eqref{eq:odpowiedz_skokowa}) obiektu w następujący sposób:
\begin{itemize}
    \item wzmocnienie $K_b$ to iloraz wartości ustalonej $h_{ss}$ oraz zadanej amplitudy $A_\text{ref}$,
    \item stała czasowa $T_b$ równa jest ilorazowi pola powierzchni $S^+$ pod asymptotą i nad wykresem charakterystyki oraz wartości ustalonej.
\end{itemize}

\begin{equation}
    h(t) = h_{ss} \left( 1 - e^{-\frac{t}{T_b}} \right) \label{eq:odpowiedz_skokowa}
\end{equation}

Pole powierzchni $S^+$ (zobrazowane na \cref{fig:odpowiedz_skokowa_obiektu_pierwszego_rzedu}) określa się jako całkę:
\begin{nalign}
    S^+ &= \int \limits_{0}^{\infty} \left(h_{ss} - h(t) \right) \mathrm{d}x \\
        &= \int \limits_{0}^{\infty} \left(K_b A_\text{ref} - K_b A_\text{ref} + K_b A_\text{ref} e^{-\frac{t}{T_b}} \right) \mathrm{d}x \\
        &= T_b K_b A_\text{ref} e^{-\frac{t}{T_b}}|_{t=0}^{\infty} \\
        &= T_b K_b A_\text{ref} \label{eq:pole_nad_figura}
\end{nalign}

Przykładową odpowiedź skokową wału motoreduktora wraz z jej aproksymacją zaprezentowano na \cref{fig:identyfikacja_belki}. W celu identyfikacji zdecydowano się wykorzystać odczyt prędkości wału motoreduktora z~dwóch powodów:
\begin{itemize}
    \item po pierwsze, tylko prędkość kątowa wału jest wartością zachowującą się ściśle jak obiekt inercyjny pierwszego rzędu,
    \item po drugie, obrót belki zależny jest sinusoidalnie od obrotu wału silnika; stąd trudniejsze byłoby odczytanie parametrów ,,ukrytych'' w przebiegu sinusoidalnym, gdyby identyfikację przeprowadzano na kącie lub prędkości kątowej belki.
\end{itemize}
Algorytm identyfikacyjny (zob. rozdział \ref{sec:ch8_algorytm_identyfikacji_belki}) również korzysta z odpowiedzi skokowej wału motoreduktora.

\begin{figure}[ht]
    \centering
    \begin{tikzpicture}
        \begin{axis}[
            domain=0.0:3,
            xmin=0, xmax=2,
            ymin=0, ymax=40,
            samples=400,
            axis y line=center,
            axis x line=middle,
            legend pos=north west,
        ]
            \addplot+[name path=A, no marks] {25*(1-e^(-x/0.118))};
            \addlegendentry{Odpowiedź skokowa $h(t)$};
            \addplot[dashed,name path=B] {25};
            \addlegendentry{Wartość graniczna (stan ustalony) $h_{ss}$};
            \addplot[pattern=north east lines,pattern color=gray] fill between[of=A and B];
            \addlegendentry{Pole nad odpowiedzią skokową $S^+$};
        \end{axis}
    \end{tikzpicture}
    \caption{Wykres odpowiedzi skokowej obiektu inercyjnego pierwszego rzędu.}
    \label{fig:odpowiedz_skokowa_obiektu_pierwszego_rzedu}
\end{figure}

\begin{figure}[ht]
    \centering
    \includesvg[width=1\textwidth,svgpath=./vector_graphics/]{identyfikacja_belki}    
    \caption{Przykład aproksymacji odpowiedzi skokowej prędkości obrotowej wału motoreduktora.}
    \label{fig:identyfikacja_belki}
\end{figure}

Należy zwrócić uwagę, że wartością, która częściej jest wykorzystywana w algorytmach sterowania w tym obiekcie, jest nie prędkość kątowa wału motoreduktora, ale jego pozycja kątowa. Rozważany w procesie identyfikacji obiekt \eqref{eq:transmitancja_obiektu_pierwszego_rzedu} nie reprezentuje odpowiedzi kątowej silnika, dlatego należy użyć dodatkowego integratora, co pozwoli uzyskać kąt obrotu:
\begin{equation}
    H_b(s) = G_b(s) \cdot \frac{1}{s} = \frac{K_b}{T_b s^2 + s} \label{eq:transmitancja_silnik_odp_katowa}
\end{equation}

Obiektowi w ten sposób opisanemu odpowiadają następujące macierze stanu\footnote{Dzięki użyciu charakterystyki obiektu całkującego rzeczywistego \eqref{eq:transmitancja_silnik_odp_katowa} możliwe jest rozszerzenie macierzy wyjścia tak, aby móc odczytać dwa stany jednocześnie; nie jest to jednak wymagane w dalszej analizie, dlatego zostało pominięte.}:
\begin{nalign}
    A_I &= \begin{bmatrix}
        0 & 1 \\ 0 & -\frac{1}{T_b}
    \end{bmatrix} \\
    B_I &= \begin{bmatrix}
        0 \\ \frac{K_b}{T_b}
    \end{bmatrix} \\
    C_I &= \begin{bmatrix}
        1 & 0
    \end{bmatrix} \\
    D_I &= 0  \label{eq:macierze_stanu_obiektu_pierwszego_rzedu}
\end{nalign}

Regulator, który steruje belką, oparty jest o stan (kąt i prędkość kątową obrotu) belki (zob. rozdział \ref{sec:ch6_regulator_belki}), a jego nastawy obliczono wykorzystując funkcję \texttt{looptune} z programu \textsc{Matlab}, która optymalizuje zadany układ w dziedzinie częstotliwości. Przy samostrojeniu tego regulatora zdecydowano się na inne podejście oparte o przesuwanie (lokowanie) wartości własnych zamkniętego układu. W tym celu wyliczono zależność algebraiczną pomiędzy nastawami regulatora a zadanymi wartościami własnymi.

Zamknięty układ regulacji z regulatorem od stanu $K_I = \begin{bmatrix}
    k_1 & k_2
\end{bmatrix}$ opisany jest równaniem:
\begin{equation}
    \dot{x} = \underbrace{(A_I - B_I K_I)}_{M \in \mathbb{R}^{2 \times 2}} x
\end{equation}

Macierz $M$ ma dwie wartości własne $\lambda_1$ oraz $\lambda_2$, które spełniają następujące zależności:
\begin{nalign}
    \det(M) &= \lambda_1 \lambda_2 \\
    \tr(M) &= \lambda_1 + \lambda_2 \label{eq:obiekt1_zaleznosc_lambd}
\end{nalign}

Dzięki \eqref{eq:obiekt1_zaleznosc_lambd} możliwe jest uzyskanie układu równań, po którego rozwiązaniu uzyskuje się następującą zależność na $K_I$:

\begin{nalign}
    k_1 &= \frac{\lambda_1 \lambda_2}{\frac{K_b}{T_b}} \\
    k_2 &= \frac{\frac{-1}{T_b} - \lambda_1 - \lambda_2}{\frac{K_b}{T_b}} \label{eq:zaleznosc_regulator_belki}
\end{nalign}

Należy zwrócić uwagę, że postać równań stanu obiektu \eqref{eq:macierze_stanu_obiektu_pierwszego_rzedu} nie do końca odpowiada postaci otrzymanej z linearyzacji \eqref{eq:rownania_stanu_belki} i \eqref{eq:rownania_wyjscia_belki}, w związku z tym zdecydowano się narzucić inne wartości własne macierzy $M$ niż te, które ma ,,zwinięty'' układ regulatora pozycji belki (\cref{fig:schemat_regulacji_belka}):

\begin{equation}
    \Lambda(M) = \begin{bmatrix}
    -8,2378 \\ -37,7286
    \end{bmatrix}
\end{equation}

%%%%
\section{Algorytm identyfikacji obiektu belki}
\label{sec:ch8_algorytm_identyfikacji_belki}

Opisany w rozdziale \ref{sec:ch8_identyfikacja_belki} eksperyment (odpowiedź skokowa prędkości kątowej wału motoreduktora) został zaimplementowany w sterowniku PLC, a jego wyniki służą do obliczania nowych nastaw regulatora według równań \eqref{eq:zaleznosc_regulator_belki}. Eksperyment polega na ustawieniu sterowania silnika na \SI{50}{\percent} na czas \SI{2}{\second}~i~zarejestrowaniu odczytu prędkości kątowej wału motoreduktora (jego odpowiedzi skokowej).

Procedura identyfikacyjna została przeprowadzona w sekwencji opisanej na \cref{fig:schemat_samostrojenia_belka}, a sama sekwencja została zaimplementowana w języku drabinkowym \textit{LAD} podobnie jak sekwencja bazowania czy sekwencja główna (rozdział \ref{sec:sekwencja_glowna}).

\begin{figure}[ht]
    \centering
    
    \begin{tikzpicture}[auto, node distance=1cm,>=latex']
    \node [startblock] (S1) {Reset zapisanych wartości};
    \node [block, below=2of S1] (S2) {Obliczanie całki pod wykresem};
    \node [block, right=of S2] (S3) {Zbieranie danych do obliczenia $h_{ss}$};
    \node [block, below=2of S2] (S4) {Obliczanie parametrów};
    \node [block, below=of S4] (S5) {Obliczanie regulatora};
        
    \draw [->] (S1) -- node [name=J1,left] {} node[name=T1,pos=0.75] {$T_1$} (S2);
    \draw [->] (J1) -| node [name=T2,pos=0.75] {$T_2$} (S3.north);
    \draw [->] (S2) -- node [name=J2,left] {} node[name=T3,pos=0.75] {$T_3$} (S4);
    \draw [-] (S3) |- (J2);
    \draw [->] (S4) -- (S5);
    \end{tikzpicture}
    
    \caption{Schemat sekwencji samostrojenia regulatora belki.}
    \label{fig:schemat_samostrojenia_belka}
\end{figure}

Tranzycjom $T_i$ zaznaczonym na schemacie na \cref{fig:schemat_samostrojenia_belka} odpowiadają następujące warunki:
\begin{itemize}
    \item $T_1$ -- rozpoczęcie eksperymentu (podanie sterowania \SI{50}{\percent} na silnik),
    \item $T_2$ -- upłynięcie czasu połowy eksperymentu (\SI{1}{\second}),
    \item $T_3$ -- zakończenie eksperymentu.
\end{itemize}

Krok pierwszy algorytmu, oznaczony jako \textit{Reset zapisanych wartości}, zeruje zapisaną wartość całki oraz dwa rejestry przechowujące sumę oraz ilość wartości, co uaktualniane jest co cykl procesora w~bloku \textit{Zbieranie danych do obliczenia $h_{ss}$}. Wartości te służą do obliczenia średniej arytmetycznej, która odpowiada oczekiwanej wartości ustalonej prędkości $h_{ss}$. Jest to wymuszone mocną kwantyzacją chwilowych wartości prędkości, co można zaobserwować na \cref{fig:identyfikacja_belki}.

Obliczanie całki odbywa się za pomocą bloku \texttt{LGF\_Integration}\cite{LGF} dostępnego w bibliotece \texttt{LGF} udostępnionej przez firmę \textsc{Siemens}. Na podstawie wzorów umieszczonych w rozdziale \ref{sec:ch8_identyfikacja_belki} obliczane są parametry $S$ (pole całkowite), $A_b$, $K_b$, $S^+$, $T_b$ oraz nastawy regulatora $k_1$, $k_2$.

%%%%
\section{Identyfikacja obiektu kulki}
\label{sec:ch8_identyfikacja_kulki}

Wykorzystując schemat sił działających na kulkę znajdującą się na pochylonej pod kątem $\theta$ belce (\cref{fig:sily_dzialajace_na_kulke}) wyznaczono, w sposób opisany poniżej, zależność przyspieszenia liniowego, które działa na kulkę, od kąta pochylenia belki.

\def\iangle{6} % Angle of the inclined plane
\def\arcr{1.1cm} % Radius of the arc used to indicate angles
\def\down{-90}

\begin{figure}[ht]
    \centering
    \begin{tikzpicture}[
        scale = 3,
        force/.style={>=latex,draw=blue,fill=blue},
        plane/.style={draw=black},
        M/.style={draw,circle,fill=lightgray,minimum size=1cm,thin},
        axis/.style={densely dashed,gray,font=\small},
    ]
        %\draw[plane] (0,-) coordinate (base) 
        %-- coordinate[pos=0.5] (mid) ++(\iangle:3) coordinate (top)
        %|- (base) -- cycle;
        \draw[dashed] (3,-1) -- (0,-1) coordinate (base);
        \draw (base)
        -- coordinate[pos=0.5] (mid) ++(\iangle:3) coordinate (top);
        \path (mid) node[M,rotate=\iangle,yshift=0.5cm] (M) {};
        \draw[->] (base)++(\arcr,0) arc (0:\iangle:\arcr);
        \path (base)++(\iangle*0.5:\arcr+3pt) node {$\theta$};
        
        \begin{scope}[rotate=\iangle]
            \draw [force,->] (M.center) -- ++(0,0.4) node[above right] {$\vec{N}$};
            % Assuming that Mg = 1. The normal force will therefore be cos(alpha)
            \draw [force,->] (M.west) -- ++(-0.3,0) node[above] {$\vec{F_1}$};
            \draw [force,->] (M.south) -- ++(0.3,0) node[above] {$\vec{N_1}$};
        \end{scope}
        \draw[force,->] (M.center) -- ++(0,-0.4) node[below] {$\vec{Q}$};
    \end{tikzpicture}
    
    \caption{Schemat sił działających na kulkę znajdującą się na pochylonej pod kątem $\theta$ belce.}
    \label{fig:sily_dzialajace_na_kulke}
\end{figure}

Zgodnie z rysunkiem, na kulkę działają dwie siły: $\vec{Q}$ (siła ciężkości) i $\vec{N}$ (siła reakcji podłoża). Dodatkowo zostały zaznaczone składowe równoległe do powierzchni belki wspomnianych sił, $\vec{F_1}$ oraz $\vec{N_1}$; ich wypadkowa $F = am = F_1 - N_1$ jest siłą powodującą staczanie się kulki z przyspieszeniem wypadkowym $a$ wzdłuż belki.

Wartość siły ciężkości działającej na kulkę definiowana jest jako $Q=mg$, gdzie $m$ to masa kulki, a~$g$~to przyspieszenie ziemskie. Automatycznie składowa $F_1$ ma wartość $F_1=mg\sin\theta$.

Siła $\vec{N_1}$, pochodząca od tarcia suchego kulki o belkę, wprowadza kulkę w ruch obrotowy, stąd $\epsilon J = r N_1$, gdzie $\epsilon$ to przyspieszenie kątowe kulki, $J$ to moment bezwładności kulki, a $r$ to jej promień. Dodatkowo pojawia się zależność przyspieszenia liniowego od kątowego $a = \epsilon r$, która jest właściwa przy braku poślizgu kulki o belkę.

Podstawiając wartości sił $F_1$ i $N_1$ do równania na siłę wypadkową działającą na kulkę otrzymano:

\begin{nalign}
    F &= mg\sin\theta - \frac{\epsilon}{r}J \\
    a m &= mg\sin\theta - \frac{a}{r^2}J \\
    a\left(m + \frac{J}{r^2} \right) &= mg\sin\theta \\
    a &= \frac{mg\sin\theta}{m+\frac{J}{r^2}} \\
    a &= \frac{r^2mg\sin\theta}{r^2m+J}
    \label{eq:przyspieszenie_kulki1}
\end{nalign}

Wiadomo, że moment bezwładności $J$ jednolitej kulki o promieniu $r$ wynosi $J=\frac{2}{5}r^2m$. Podstawiając ten wzór do wyniku z \eqref{eq:przyspieszenie_kulki1} otrzymano:

\begin{nalign}
    a &= \frac{r^2mg\sin\theta}{r^2m+J} = \\
      &= \frac{r^2mg\sin\theta}{r^2m+\frac{2}{5}r^2m} = \\
      &= \frac{g\sin\theta}{1+\frac{2}{5}} \\
    a &= \frac{5}{7} g \sin\theta
    \label{eq:przyspieszenie_kulki2}
\end{nalign}

Wartość $\frac{5}{7}g$ to w przybliżeniu 7, natomiast dla małych kątów $\sin\theta \approx \theta$. Stąd ostateczny wynik:

\begin{equation}
    a \approx 7 \cdot \theta \label{eq:przyspieszenie_kulki3}
\end{equation}

Oznacza to, że przyspieszenie kulki staczającej się po pochylonej belce jest zależne wyłącznie od kąta pochylenia belki. Należy w tym miejscu zwrócić uwagę na wyniki linearyzacji \eqref{eq:rownania_stanu_kulki2} oraz \eqref{eq:rownania_wyjscia_kulki2}, które zostały przypomniane poniżej:

\begin{align*}
    \begin{bmatrix}
        \dot{x}_1 \\ \dot{x}_2
    \end{bmatrix}
    &= \begin{bmatrix}
        0 & 1 \\
        0 & 0
    \end{bmatrix}
    \begin{bmatrix}
        x_1 \\ x_2
    \end{bmatrix}
    +
    \begin{bmatrix}
        0 & 0 \\ 7,0047 & 0
    \end{bmatrix}
    \begin{bmatrix}
        u_1^* \\ u_2^*
    \end{bmatrix}\\
    \begin{bmatrix}
        y_1 \\ y_2
    \end{bmatrix}
    &= \begin{bmatrix}
        1 & 0 \\
        0 & 1 \\
    \end{bmatrix}
    \begin{bmatrix}
        x_1 \\ x_2
    \end{bmatrix}
    + \begin{bmatrix}
        -0,0675 & 0 \\
        0 & -0,0675
    \end{bmatrix}
    \begin{bmatrix}
        u_1^* \\ u_2^*
    \end{bmatrix}
\end{align*}

Jak widać z równań stanu, $x_1$ jest całką $x_2$, a pochodna stanu $x_2$ --- $\dot{x}_2$ --- jest siedmiokrotnością sterowania $u_1^*$. Wyniki te są zgodne z otrzymaną zależnością \eqref{eq:przyspieszenie_kulki3}, gdyż $x_1$ to pozycja liniowa kulki, $x_2$ to jej prędkość liniowa, $\dot{x}_2$ to przyspieszenie liniowe, a $u_1^*$ to kąt belki.

Powyższy wywód dowodzi, że przyspieszenie kulki, które charakteryzuje jej dynamikę, jest niezależne od momentu bezwładności kulki --- czyli od jej masy i rozmiaru --- a~zatem eksperymenty identyfikacyjne (takie jak pomiar czasu i/lub drogi staczania się kulki) nie dadzą rezultatów, które mogłyby w~jakikolwiek sposób pomóc w doborze regulatora.

Regulator nadrzędny został zbudowany dla systemu ,,zwiniętego'' (zob. rozdział \ref{sec:ch6_regulator_kulki}), obejmującego wewnętrzną pętlę sprzężenia zwrotnego dla systemu belki oraz system kulki, a zatem po zmianie parametrów tej pętli wewnętrznej można ponownie przeprowadzić obliczenia dające nowe nastawy regulatora nadrzędnego. Po przeprowadzeniu takich obliczeń z wykorzystaniem funkcji \texttt{looptune} okazało się, że ,,nowe'' nastawy są zbliżone do nastaw regulatora dla modelu zlinearyzowanego, dlatego zdecydowano się pominąć te obliczenia w procesie samostrojenia.

%%%%
\section{Podsumowanie}

W niniejszym rozdziale zaprezentowano metodę samostrojenia regulatora podrzędnego opartą o~identyfikację modelu matematycznego obiektu inercyjnego pierwszego rzędu, jakim jest silnik użyty w obiekcie regulacji. Opisano również matematyczne zależności między parametrami takiego modelu a~jego odpowiedzią skokową. Następnie przeprowadzono obliczenia zależności pomiędzy regulatorem od stanu, obiektem inercyjnym pierwszego rzędu oraz zadanymi wartościami własnymi układu z zamkniętą pętlą sprzężenia zwrotnego. W kolejnej części rozdziału opisano implementację procedury samostrojenia, wykorzystującą zależności parametrów obiektu i odpowiedzi skokowej, na sterowniku PLC.

Na koniec przeprowadzono wywód udowadniający, że przeprowadzona identyfikacja kulki nie powiedzie się, gdyż jej model liniowy jest niezależny od parametrów fizycznych kulki, a jedynie od kąta obrotu belki. Z tego powodu zaniechano implementacji procedury samostrojenia dla kulki.

%---------------------------------------------------------------------------
\chapter{Symulacje i eksperymenty}
\label{cha:ch9_symulacje_i_eksperymenty}

Lorem ipsum.

%---------------------------------------------------------------------------
\chapter{Wnioski}
\label{cha:ch10_wnioski}

Lorem ipsum.

%---------------------------------------------------------------------------

\appendix
\chapter{Warianty zespołu napędowego}
\label{appA_warianty_zespolu_napedowego}

Cechą charakterystyczną obiektu regulacji typu kulka i belka jest pewna dowolność w wyborze układu napędowego, wprawiającego belkę w ruch obrotowy. W niniejszym dodatku przeglądnięto kilka alternatywnych sposobów, które były brane pod uwagę przy projektowaniu układu.

\section{Siłownik liniowy}

Oddziałujący na koniec belki siłownik musiałby zostać zamontowany pionowo; sprzęgnięcie go z belką również nie byłoby trywialne, gdyż punkt na końcu belki zakreśla w przestrzeni łuk, a nie odcinek (tak jak na przykład tłok w klasycznym mechanizmie korbowym). Kolejną wadą jest wysoka cena i prawdopodobne konieczne użycie specjalistycznego sterownika.

\section{Serwomechanizm modelarski}

Rozwiązanie stosowane w niektórych układach typu kulka i belka. Cechuje je przystępna cena, ale również brak możliwości ingerencji w wewnętrzny układ regulacji położenia wału.

Zastosowanie takiego silnika nie rozwiązuje problemu przeniesienia napędu.

\section{Silnik tarczowy (\textit{pancake}) z wirnikiem PCB}

Silnik tarczowy DC z wirnikiem wykonanym w technologii obwodów drukowanych jest propozycją wymagającą zasilacza dużej mocy oraz oddzielnie zamontowanego enkodera; ma też wysoką cenę.

\section{Silnik krokowy}
\chapter{Alternatywne czujniki pozycji kulki}
\label{appB_alternatywne_czujniki_pozycji_kulki}

W systemach typu kulka i belka, poza dowolnością w wyborze mechanizmu napędu, istnieje kilka różnych alternatywnych czujników mierzących położenie kulki na belce.

\section{Listwa rezystancyjna}

% TODO: opisz to lepiej bo słabo
Czujnik zastosowany na zaprezentowanym urządzeniu komercyjnym (\cref{fig:quanser_ball_beam}). Idea działania jest owiana tajemnicą.

Wadami takiego rozwiązania jest ograniczenie do wąskiego zakresu rozmiarów kulek oraz wyłącznie do kulek przewodzących prąd elektryczny.

\section{System wizyjny}

Użycie kamery do śledzenia pozycji kulki jest możliwe, ale w celu uproszczenia algorytmów kamera powinna być umieszczona w tym samym układzie odniesienia, co belka. Oznacza to, że wraz z ruchem obrotowym belki, również kamera powinna się obracać, co stanowi spore wyzwanie konstrukcyjne.

Dodatkowo zastosowanie taniej kamery (nieprzemysłowej) wymaga osobnego komputera kontrolującego algorytmy wizyjne; jest to zatem droga pozycja.

\section{Laserowy czujnik odległości}

Przemysłowe czujniki odległości są głównie czujnikami laserowymi. Zapewniają doskonałą dokładność, jest to jednak okupione dużą ceną i koniecznością dobrego wycelowania punktu lasera w przedmiot, do którego odległość jest mierzona. W przypadku metalicznej kulki, trafienie laserem niecentralnie w kulkę może powodować problem z prawidłowym odczytem odległości.

\section{Ultradźwiękowy czujnik odległości}

W przypadku rozwiązań hobbystycznych, ultradźwiękowe czujniki odległości potrafią być bardzo tanie, ale niedostatecznie wydajne lub dokładne. Istnieją czujniki przemysłowe tego typu, ale ich cena jest duża. Dodatkowym problemem jest niewielka powierzchnia celu (kulki), w który fale dźwiękowe muszą trafić. Mogłoby to skutkować fałszowaniem pomiaru, np. w przypadku odbijania się fal od ścian belki.

%\section{Liniowy enkoder magnetyczny}
%
%Przemysłowy enkoder magnetyczny jest czujnikiem bardzo dokładnym i drogim. Podobnie jak w przypadku listwy rezystancyjnej, wymaga on kulek wykonanych z odpowiedniego tworzywa.

\printbibliography
\listoffigures
\listoftables

\end{document}
