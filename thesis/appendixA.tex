\chapter{Warianty zespołu napędowego}
\label{appA_warianty_zespolu_napedowego}

Cechą charakterystyczną obiektu regulacji typu kulka i belka jest pewna dowolność w wyborze układu napędowego, wprawiającego belkę w ruch obrotowy. W niniejszym dodatku przeglądnięto kilka alternatywnych sposobów, które były brane pod uwagę przy projektowaniu układu.

\section{Siłownik liniowy}

Oddziałujący na koniec belki siłownik musiałby zostać zamontowany pionowo; sprzęgnięcie go z belką również nie byłoby trywialne, gdyż punkt na końcu belki zakreśla w przestrzeni łuk, a nie odcinek (tak jak na przykład tłok w klasycznym mechanizmie korbowym). Kolejną wadą jest wysoka cena i prawdopodobne konieczne użycie specjalistycznego sterownika.

\section{Serwomechanizm modelarski}

Rozwiązanie stosowane w niektórych układach typu kulka i belka. Cechuje je przystępna cena, ale również brak możliwości ingerencji w wewnętrzny układ regulacji położenia wału.

Zastosowanie takiego silnika nie rozwiązuje problemu przeniesienia napędu.

\section{Silnik tarczowy (\textit{pancake}) z wirnikiem PCB}

Silnik tarczowy DC z wirnikiem wykonanym w technologii obwodów drukowanych jest propozycją wymagającą zasilacza dużej mocy oraz oddzielnie zamontowanego enkodera; ma też wysoką cenę.

\section{Silnik krokowy}